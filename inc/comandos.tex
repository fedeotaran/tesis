% Comandos personalizados

% {\fechaPresentacion} :: para escribir la fecha de presentación del trabajo
\newcommand{\fechaPresentacion}{\today}
% {\unlp} :: para escribir "Universidad Nacional de La Plata"
\newcommand{\unlp}{Universidad Nacional de La Plata}
% {\facultad} :: para escribir "Facultad de Informática"
\newcommand{\facultad}{Facultad de Informática}
% {\cespi} :: para escribir "CeSPI"
\newcommand{\cespi}{CeSPI}
% {\direccionDesarrollo} :: Para escribir "Dirección de Desarrollo del CeSPI"
\newcommand{\direccionDesarrollo}{Dirección de Desarrollo del {\cespi}}
% {\tituloTrabajo} :: Para escribir el título de la tesina
\newcommand{\tituloTrabajo}{Propuesta de una solución de Monitoreo para sistemas del {\cespi}.}
% \tituloTrabajoDosLineas :: Para escribir el título de la tesina en dos líneas (carátula)
\newcommand{\tituloTrabajoDosLineas}{Propuesta de una solución de Monitoreo\\* para sistemas del {\cespi}}
% {\fedeotaran} :: para escribir "Federico Otarán"
\newcommand{\fedeotaran}{Federico Otarán}
% {\otaranfede} :: para escribir "Otarán, Federico"
\newcommand{\otaranfede}{Otarán, Federico}
% {\nicaperera} :: para escribir "Nicanor Perera"
\newcommand{\nicaperera}{Nicanor Perera}
% {\otaranfede} :: para escribir "Otarán, Federico"
\newcommand{\pereranica}{Perera, Nicanor}

% \eng{English expression} :: para denotar que "English expression" está en inglés
\newcommand{\eng}[1]{\textit{#1}}

% {\caratula} :: para generar la carátula de la tesina
\newcommand{\caratula}{
  \begin{center}
    \includegraphics{src/images/caratula/unlp.png}\\
    \huge{\unlp}\\
    \vspace{5mm}
    \huge{\facultad}\\
    \vspace{5mm}
    \large{Tesina de la Licenciatura}\\
    \vspace{15mm}
    \huge{\tituloTrabajoDosLineas}\\
    \vspace{10mm}
    \large{\textbf{\pereranica} \\
    \textbf{\otaranfede}}\\
    \vspace{20mm}
    \large{Director: Luengo, Miguel}\\
    \large{Asesor Profesional: Rodriguez, Christian Adrián}\\
    \vspace{20mm}
    \normalsize{\fechaPresentacion}\\
  \end{center}
}

\addto\captionsspanish{%
  \renewcommand\appendixname{Anexo}
  \renewcommand\appendixpagename{Anexos}
  \renewcommand\appendixtocname{Anexos}
}

\addto\extrasspanish{%
  \def\sectionautorefname{capítulo}%
  \def\subsectionautorefname{sección}%
  \def\subsubsectionautorefname{sección}%
  \renewcommand{\appendixautorefname}{anexo}%
}

% {\checkmark} :: para imprimir un check (tilde)
\def\checkmark{\tikz\fill[scale=0.4](0,.35) -- (.25,0) -- (1,.7) -- (.25,.15) -- cycle;}
