No existe, en el ambiente de la informática, un acuerdo común sobre la
definición precisa de monitoreo. Es por esta razón que en la \autoref{monitoreo}
cotejaremos y analizaremos las opiniones de varios autores sobre el significado
de este término. A partir de estas definiciones intentaremos evidenciar la
importancia de su aplicación en proyectos de software.

Para poder comprender algunos de los pasajes de los capítulos siguientes y
brindar la base teórica de este trabajo, en la \autoref{metricas-y-timeseries}
nombraremos las definiciones de algunos términos importantes, incluyendo
métricas, series de tiempo, estadísticos, intervalos de tiempo, granularidad,
percentiles y distribución de frecuencias, entre otros.

El proceso de monitoreo comienza con la acumulación de datos por agentes de
recolección. En la sección \autoref{recoleccion-de-datos} daremos la definición
de agente de recolección, y enumeraremos los distintos tipos de agentes que
existen. Además indagaremos en cómo las necesidades de diferentes tipos de
usuario determinan los datos a recolectar.

Se denomina log al registro de una operación en una computadora, usualmente
almacenado en un archivo. En la \autoref{utilidad-de-los-logs} profundizaremos
en la definición de log y explicaremos su utilidad para el monitoreo.
Explicaremos las partes de un log, su formato y las dificultades más comunes a
la hora de manipular este tipo de registros.

La visualización de datos es el proceso de interpretación, contrastación y
comparación de datos que ofrece un conocimiento en profundidad y detalle de
los mismos de tal forma que se transformen en información comprensible para las
personas. En la \autoref{visualizacion-efectiva} ahondaremos más en este tema.

Finalmente, en la \autoref{alertas} de este capítulo definiremos alertas en el
contexto de monitoreo, manifestaremos la importancia de las mismas y
describiremos los obstáculos más comunes a la hora de implementar una solución
de este tipo.
