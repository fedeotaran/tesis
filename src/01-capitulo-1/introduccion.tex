No existe, en el ambiente de la informática, un acuerdo común sobre la definición precisa de monitoreo. Es por esta razón que en la sección 1.1 compararemos y analizaremos las opiniones de varios autores sobre el significado de este término. A partir de estas definiciones intentaremos evidenciar la importancia de su aplicación en proyectos de software.

Para poder comprender algunos de los pasajes de los capítulos siguientes y brindar la base teórica de este trabajo, en la sección 1.2 trataremos de definir algunos términos importantes, incluyendo métricas,  time-series, estadísticos, intervalos de tiempo, granularidad, percentiles y distribución de frecuencias, entre otros. 

El proceso de monitoreo comienza con la recolección de datos por agentes de recolección. En la sección 1.3 daremos la definición de agente de recolección, y enumeraremos los distintos tipos de agentes que existen. Además indagaremos en cómo las necesidades de diferentes tipos de usuario determinan los datos a recolectar.

Se denomina log al registro de una operación en una computadora, usualmente almacenado en un archivo. En la sección 1.4 profundizaremos en la definición de logs y explicaremos su utilidad para el monitoreo. Explicaremos las partes de un log, el formato de archivos de logs y las dificultades más comunes a la hora de manipular este tipo de registros.

La visualización de datos es el proceso de interpretación, contrastación y comparación de datos que permite un conocimiento en profundidad y detalle de los mismos de tal forma que se transformen en información comprensible para las personas. En la sección 1.5 ahondaremos más en este tema.

Finalmente, utilizaremos la sección 1.6 de este capítulo para definir alertas en un contexto de monitoreo, manifestar la importancia de las mismas y describir los obstáculos más comunes a la hora de implementar una solución de alertas.
