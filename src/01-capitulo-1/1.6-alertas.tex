\subsection{Alertas}
\label{alertas}

En las secciones anteriores se analizaron conceptos importantes en cuanto a
la obtención, manipulación, procesamiento, almacenamiento, resumen y
visualización de información de sistemas que proveen información valiosa.

En el libro Effective Monitoring and Alerting, el autor explica que con la
información generada a través de técnicas de monitorización es posible producir
un gran número de tableros a los cuales es importante prestarles atención.
Contratar personas para visualizar estos tableros no siempre es una tarea
rentable. Esta tarea no es muy gratificante, e incluso si lo fuera, es por lo
menos discutible que un operador humano sea mejor que una máquina o algoritmo
reconociendo patrones cuya atención pueda prevenir una situación no deseada
\cite[p. ~ 47]{monitoreo:efective_monitoring_and_alerting}.

Una alerta puede ser un mensaje enviado con el fin de notificar sobre un evento
importante a una persona. Este mensaje puede ser transmitido mediante correo
electrónico, servicio de mensajes simples, mensaje instantáneo, llamada
telefónica o a través de un servicio de notificaciones de alguna herramienta de
\eng{software}.

Es deseable que un sistema de monitoreo tenga la capacidad de detectar aquellos
eventos significativos o que denoten un grave cambio de estado en el negocio,
aplicación o sistema, y puedo notificarlo a los interesados a través de un
sistema de alertas.

Los sistemas de alertas suelen ser configurados por los profecionales de
\gls{acro:it} para detectar y prevenir problemas. Las alertas pueden brindar
información sobre eventos no deseados, umbrales sobrepasados, caídas del
sistema o hitos alcanzados.

Las alertas deben ser transmitidas al destinatario adecuado, es decir, a una
persona que esté obligada a tratar con el evento.

Según Jason Dixon, autor de Monitoring with Graphite, para todos los propósitos
prácticos, alertar constituye el momento en que un sistema de monitoreo
identifica un fallo que requiere alguna acción adicional. En la mayoría de los
casos, el sistema está diseñado para enviar un correo electrónico a un miembro
del equipo de operaciones, pero conseguir que la alerta llegue a su destino no
suele ser tan simple como parece.

Por ejemplo, puede suceder que el destinatario esté fuera del rango de su
servicio celular, o que tenga la computadora apagada. Es por esto que la
programación de llamadas y el enrutamiento de notificaciones deben ser
incluidos en cualquier discusión de alertas.

A medida que una organización crece, es mayor el número de personas que pueden
resolver el problema notificado, pero, al mismo tiempo, la complejidad para
saber cómo planificar estas alertas aumenta.

En este punto entran en juego las herramientas de manejo de alertas. Las mismas
se encargan de administrar alertas y gestionar los avisos. Algunas incluso
permiten organizar un calendario de llamadas \cite[p. ~
17]{monitoreo:monitoring_with_grapfite}.

Las alertas son registradas a menudo en la forma de una petición o \eng{ticket}
en un sistema de \gls{term:its} (\eng{Issue Tracking System}).

Un \gls{term:its} es un paquete de \eng{software} que administra y mantiene
listas de incidentes conforme son requeridos por una institución. Estos
sistemas suelen ser usados por personal de servicio al cliente para crear,
actualizar y resolver incidentes reportados por usuarios, empleados de la
organización o sistemas de alertas automatizados.

Stawek Ligus aconseja operar de acuerdo al principio de monitoreo extensivo y
de alerta selectiva. Esto significa identificar qué métricas son útiles para el
negocio y configurar alarmas para las series temporales a partir de indicadores
clave. El autor sugiere una lista de métricas a considerar como candidatas para
alertar:

\begin{itemize}
  \item Recursos:
    \subitem \textbf{Retardo de red y pérdida de paquetes}: para obtener la
    métrica de latencia se puede utilizar el comando \texttt{\gls{term:ping}}
    tanto a una dirección externa como a una dirección interna y registrar el
    tiempo de ida y vuelta de cada una. Además se puede utilizar la información
    de \texttt{\gls{term:ping}} para calcular la pérdida de paquetes.  Si el
    paquete llega con éxito se anota 0, y si no llega se anota 1. Calculando el
    promedio de estos valores se obtiene la pérdida de paquetes en términos
    porcentuales.
    \subitem \textbf{Utilización de \gls{acro:cpu}}: este valor sirve como
    indicador para el esfuerzo computacional de las operaciones. En un sistema
    \gls{term:linux} puede obtenerse analizando el directorio
    \texttt{/proc/stat}.
    \subitem \textbf{Espacio disponible de disco y memoria}: es importante
    evitar que el almacenamiento local se llene y que la cantidad de espacio
    libre en disco se aproxime a los límites. El uso excesivo del
    almacenamiento local puede inducir a un comportamiento impredecible para
    las aplicaciones.

  \item Plataforma:
    \subitem \textbf{Tiempo de respuesta}: puede ser útil extraer y registrar
    las estadísticas de tiempo de respuesta de los logs de las aplicaciones.
    Particularmente puede ser importante prestarle atención a los cambios en el
    promedio y a los \glspl{term:percentil}.
    \subitem \textbf{Códigos de respuesta}: registrar los códigos de errores
    \gls{acro:http} en aplicaciones \eng{web} es una buena idea si se
    quiere alertar sobre una proporción inusual de códigos \gls{acro:http}
    referidos a requerimientos erróneos o problemas del servidor.

  \item Aplicación:
    \subitem \textbf{Disponibilidad}: si se quiere lanzar una alerta cuando el
    porcentaje de solicitudes exitosas se encuentre debajo de un determinado
    umbral, es buena idea configurar controles de disponibilidad desde
    distintas ubicaciones, emitir solicitudes de prueba a cada minuto y
    registrar casos de éxito y de falla.
    \subitem \textbf{Tasa de error}: si se define de forma correcta aquello que
    constituye un error en las aplicaciones y se lo monitoriza muy de cerca, se
    puede decidir lanzar una alerta cuando estos errores suceden en una
    proporción relativamente alta.

\end{itemize}

\subsubsection*{Falsas alarmas}
\label{falsas_alarmas}

Un falso positivo es una alerta desencadenada por accidente o una mala
configuración. Estas alertas pueden ser molestas para los desarrolladores, y
además pueden conducirlos a desconfiar del sistema de monitoreo.

Si un programador tiene muchos falsos positivos, puede ocurrir que reste
importancia a las alertas y pierda información importante. Por ejemplo, un
sistema de alertas que envía gran cantidad de correos electrónicos con falsos
positivos puede hacer que el usuario desestime las alertas, deje de prestarles
atención y en consecuencia no actúe cuando ocurra un evento significativo que
necesite su atención inmediata.

Un falso negativo es una alerta que nuestro sistema de monitoreo no puede
detectar. Puede ser causada por el uso de umbrales inadecuados, por falta de
controles o por la utilización de intervalos de verificación de mucha o de muy
poca duración.

Los falsos negativos suelen ser identificados demasiado tarde, resultando por
ejemplo en la caída de un sistema o en la discontinuidad de la disponibilidad
de un servicio \cite[p.~16]{monitoreo:monitoring_with_grapfite}.

Es muy importante, a la hora de armar una solución de alertas, tener muy
presenta estas situaciones y reducir al máximo el número tanto de falsos
positivos como negativos. Esto puede ser determinante para conseguir el éxito o
el fracaso de la solución propuesta.
