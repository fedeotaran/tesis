\subsection{Alertas}
\label{alertas}
Una alerta es un mensaje enviado con el fin de notificar sobre un evento importante a una persona. Este mensaje puede ser transmitido mediante correo electrónico, servicio de mensajes simples, mensaje instantáneo, llamada telefónica o a través de un servicio de notificaciones de alguna herramienta de software.

Es deseable que un sistema de monitoreo tenga la capacidad de detectar aquellos eventos significativos o que denoten un grave cambio de estado en el negocio, aplicación o sistema, y puedo notificarlo a los interesados a través de un sistema de alertas.

Los sistemas de alertas suelen ser configurados por los operarios para detectar y prevenir problemas. Las alertas pueden brindar información sobre eventos no deseados, umbrales sobrepasados, caídas del sistema o hitos alcanzados.

Las alertas deben ser transmitidas al destinatario adecuado, es decir, a una persona que esté obligada a tratar con el evento.

Las alertas son registradas a menudo en la forma de un ticket en un sistema de seguimiento de incidentes (Issue Tracking System).

Un sistema de seguimiento de incidentes es un paquete de software que administra y mantiene listas de incidentes conforme son requeridos por una institución. Estos sistemas suelen ser usados por personal de servicio al cliente para crear, actualizar y resolver incidentes reportados por usuarios, empleados de la organización o sistemas de alertas automatizados \cite[p~.2]{monitoreo:efective_monitoring_and_alerting}.

\subsubsection*{Falsas alarmas}
\label{falsas_alarmas}

Un falso positivo es una alerta desencadenada por accidente o una mala configuración. Estas alertas pueden ser molestas para los desarrolladores, y además pueden conducirlos a desconfiar del sistema de monitoreo.

Si un programador tiene muchos falsos positivos, puede ocurrir que reste importancia a las alertas y pierda información importante. Por ejemplo, un sistema de alertas que envía gran cantidad de correos electrónicos con falsos positivos puede hacer que el usuario desestime las alertas, deje de prestarles atención y en consecuencia no actúe cuando ocurra un evento significativo que necesite su atención inmediata.

Un falso negativo es una alerta que nuestro sistema de monitoreo no puede detectar. Puede ser causada por el uso de umbrales inadecuados, por falta de controles o por la utilización de intervalos de verificación de mucha o de muy poca duración.

Los falsos negativos suelen ser identificados demasiado tarde, resultando por ejemplo en la caída de un sistema o en la discontinuidad de la disponibilidad de un servicio \cite[p.~16]{monitoreo:monitoring_with_grapfite}.
