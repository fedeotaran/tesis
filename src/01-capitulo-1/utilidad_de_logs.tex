\subsection{Utilidad de los logs}
\label{utilidad_de_los_logs}
Se denomina log al registro de una operación en una computadora. Los logs son
normalmente guardados en un archivo. Los logs se utilizan para describir un
conjunto de acontecimientos, tales como: accesos de usuarios a un sistema,
manipulación de datos, auditoría, diagnóstico de dispositivos y medidas de
seguridad.

Un registro de log contiene normalmente una marca temporal, la fuente de la
información y el mensaje o contenido. La marca temporal, también conocida como
timestamp, es una secuencia de caracteres que indica la hora y fecha en la que
ocurrió un evento. La fuente es la aplicación, programa o sistema que generó el
mensaje. Muchas veces la fuente es representada por la \gls{acro:ip} o el
nombre de un \gls{term:host} y el nombre de la aplicación.

Los logs pueden ser de gran utilidad para visualizar información de recursos
del sistema, de usuarios de las aplicaciones y del funcionamiento de las mismas
aplicaciones. Sin embargo, los logs como fuente de información suelen ser
subestimados por desarrolladores y los profecionesles de \gls{acro:it}.

A menudo los logs son completamente ignorados y solamente tomados en cuenta en
situaciones de problemas en las aplicaciones o el sistema, y normalmente se los
elimina sin haber sido revisados antes
\cite[p.~16]{monitoreo:logging_and_log_management}.

Los logs pueden ser catalogados según su utilidad de varias maneras. Cada
aplicación define como clasificar los logs según su relevancia. Por ejemplo,
Syslog, el estandar de más popular para el envío de mensajes de registro en
redes, define los siguientes niveles de gravedad para sus logs
\footnote{\url{https://tools.ietf.org/html/rfc3164}}:

\begin{table}[h!]
  \begin{tabular*}{\textwidth}{ @{\extracolsep{\fill}} | c | l | l | }
    \hline
    \textbf{Cód.} & \textbf{Gravedad} & \textbf{Descripción}                      \\ \hline
    0             & Emergency         & El sistema se encuentra inutilizable      \\ \hline
    1             & Alert             & Una acción debe ser tomada inmediatamente \\ \hline
    2             & Critical          & Condición crítica                         \\ \hline
    3             & Error             & Condición de error                        \\ \hline
    4             & Warning           & Condición de advertencia                  \\ \hline
    5             & Notice            & Condición normal pero significativa       \\ \hline
    6             & Informational     & Mensaje informativo                       \\ \hline
    7             & Debug             & Mensajes de bajo nivel                    \\ \hline
  \end{tabular*}
  \caption{Tabla de tipos de logs para Syslog.}
  \label{logs_syslogs:tabla}
\end{table}

Si bien no todas las herramientas que implementan logs utilizan todos lo
niveles de severidad anteriormente mencionados, la mayor parte de ellos toman
esta tabla como referencia.

A continuación explicaremos algunos de estos niveles de relevancia:

\begin{itemize}
  \item Los logs de información proveen evidencia documentada de una secuencia
    de actividades que afectan en algún momento a una operación específica,
    procedimiento o evento. Permiten a los desarrolladores y administradores
    advertir la ocurrencia de un evento benigno, como por ejemplo el inicio de
    sesión de un usuario en un sistema.

  \item Los registros de depuración son generados por los desarrolladores con
    la finalidad de entender la actividad de un sistema, identificar problemas
    en el funcionamiento del código de la aplicación y encontrar soluciones
    para dichos problemas.

  \item Los mensajes de advertencia indican comportamientos inesperados o
    situaciones que deben ser vigiladas por un programador pero que no tienen
    un impacto negativo en el funcionamiento normal del sistema. Un ejemplo es
    mensaje aconsejando la descontinuación del uso de una funcionalidad que
    será eliminada en futuras versiones de un software.

  \item La etiqueta de error se usa para transmitir fallos que se producen en
    un sistema informático, y que requieren la atención inmediata de un
    programador o persona especializada. Indican que existe la posibilidad de
    que parte de la aplicación no esté funcionando
    \cite[p.~3]{monitoreo:logging_and_log_management}.
\end{itemize}

Es conveniente que los desarrolladores hagan buen uso de los niveles de
severidad de los logs para beneficiarse de la información que estos puedan
brindar acerca de las aplicaciones o sistemas. Si las etiquetas de relevancia
son utilizadas correctamente, se logran datos de mejor calidad que facilitan el
análisis posterior de los registros.

\subsubsection*{Retos del manejo de logs}
\label{retos_del_manejo_de_logs}

Existen varios desafíos que la mayoría de las organizaciones deben enfrentar a
la hora de gestionar los logs.

Uno de estos desafíos es la existencia de numerosas fuentes de software
generadores de registros de log. Muchas veces estas fuentes se encuentran
ubicadas en diferentes \glspl{term:host}, y una fuente puede generar diferentes
tipos de registros, por ejemplo almacenar los intentos de autenticación en una
aplicación en un archivo de logs y la información relacionada con la ocupación
de la red en otro archivo distinto.

Otro reto que presenta el manejo de logs es el hecho de que los logs registran
a menudo sólo las piezas de información que se consideran más importantes para
un evento particular. Esto puede dificultar analizar la correlación de
eventos registrados por diferentes fuentes de logs, ya que pueden no tener
valores comunes registrados.

Estas diferencias pueden ser leves, como la utilización de formato de fechas
\texttt{MMDDYYYY} en un archivo de logs y el uso de \texttt{MM-DD-YYYY} en
otro. Pero también pueden ser complejas, como la identificación de un servicio
por el nombre en un log, y la identificación de un servicio por el número de
puerto en otro.

Otro obstáculo en la gestión de logs es la inconsistencia de las marcas
temporales. Cada host que genera logs suele hacer referencia a su reloj
interno. Si el reloj de un host es inexacto, el timestamp que genera en los
logs también lo es.

Esto puede dificultar el análisis de los logs, particularmente cuando se
analizan logs de varias computadoras. Por ejemplo podría suceder que la marca
temporal de un evento A indique que dicho evento haya ocurrido 45 segundos
antes que un evento B, cuando en realidad el evento B haya ocurrido varios
minutos antes que el evento A.

Otra complicación en la administración de logs es que distintas fuentes
utilicen diferentes formatos para almacenar sus logs. Por ejemplo, algunas
herramientas pueden guardar sus logs como archivos de texto separados por comas
o por tabs, utilizando el formato Syslog o a través de \gls{acro:snmp}. También
pueden existir sistemas que almacenen logs con formato \gls{acro:xml},
\gls{acro:json} o como archivos binarios.

Algunos registros de logs han sido diseñados para ser fácilmente leídos por
seres humanos, y otros para ser fácilmente analizados por computadoras. Existen
programas que usan formatos estándares para almacenar logs, y también existen
otros que usan configuraciones propietarias. Además hay logs que no han sido
diseñados para su almacenamiento local en un archivo, sino para ser
transmitidos a un sistema de procesamiento, como por ejemplo los \eng{routers}
y \eng{switches} que implementan el protocolo
\gls{acro:snmp}\cite[p.~23]{monitoreo:log_management_guide} anteriormente
mensionado.

La naturaleza distribuida de los logs, el uso de formatos inconsistentes y el
considerable volumen de información almacenado en los mismos consiguen que la
gestión de la creación y almacenamiento de logs sea un desafío.

Los administradores de las redes y sistemas son en la mayoría de las
organizaciones los responsables de realizar el análisis de logs.

Este análisis es tratado a menudo como una tarea de baja prioridad por parte de
los administradores, y la razón de esto es que otras funciones como la solución
de problemas operativos y la atención a vulnerabilidades de seguridad suelen
requerir respuestas rápidas. Además, los administradores generalmente no
cuentan con herramientas eficaces que permitan automatizar el proceso de
análisis.

Los logs pueden ser realmente muy útiles si se los gestiona de manera correcta.
Por ejemplo, pueden describir excepcionalidades que ocurran en nuestras
infraestructuras, permitir el análisis del rendimiento de las aplicaciones y
detectar errores en el sistema. También pueden ser una buena fuente de
información para determinar las causas de un incidente.

Además de describir un problema, los logs pueden ser beneficiosos para
descubrir si se han tomado buenas decisiones de diseño de las aplicaciones y
prevenir problemas futuros. Asimismo, un log puede contener información
importante para generar alertas. En concreto, si los logs de un servidor
\eng{web} retorna un error \gls{acro:http} con código 500, es porque una
aplicación no está funcionando correctamente
\cite[p.~30-31]{monitoreo:logging_and_log_management}.

