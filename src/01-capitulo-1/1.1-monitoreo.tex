\subsection{Monitoreo}
\label{monitoreo}

En su libro Effective monitoring and Alerting for Web Operations, Stawek Ligus
define al monitoreo como el proceso de mantener en constante observación la
existencia y magnitud de los cambios de estado y el flujo de datos de un
sistema, y que tiene como objetivo identificar los fallos y ayudar en su
posterior eliminación\cite[p.~2]{monitoreo:efective_monitoring_and_alerting}.

Según el autor, las técnicas utilizadas en el monitoreo de los sistemas, son
compartidas con aquellas técnicas usadas en los campos de procesamiento de
datos en tiempo real, estadísticas y análisis de datos.

El escritor a su vez define sistema de monitoreo como el conjunto de
componentes de \eng{software} utilizados para la recolección de datos, su
procesamiento y su presentación.

En el libro Art of monitoring, James Turnbull explica de forma sencilla el
proceso por el cual el monitoreo traduce las métricas tomadas de diferentes
fuentes en experiencia de usuario medible. Esta experiencia de usuario
proporciona retroalimentación a la empresa u organización para ayudar a que la
entrega se transforme verdaderamente en lo que los clientes quieren, y también
a los desarrolladores y encargados del monitoreo indicando si algún componente
del sistema no funciona o si la calidad del producto final no es aceptable
\cite[p.~8]{monitoreo:art_of_monitoring}.

El autor resalta la importancia de la recolección, procesamiento y análisis de
los datos en un sistema y explica de forma simplificada lo mencionado
anteriormente al manifestar que un sistema de monitoreo se encarga de traducir
las métricas generadas por los sistemas y aplicaciones en información de valor
para el negocio.

Jason Dixon define, en su libro Monitoring with Graphite, al monitoreo como el
conjunto de software y procesos que son utilizados para asegurar la
disponibilidad y salud de uno o más sistemas y servicios, y en un sentido
abstracto afirma que puede ser dividido en tres grandes categorías: detección
de fallas, producción de alertas y planificación de crecimiento.

La detección de fallas consiste en identificar cuando un recurso o colección de
recursos deja de estar disponible o disminuye su rendimiento. Se suelen usar
umbrales para identificar variaciones importantes en el comportamiento del
sistema. Hoy en día además se utilizan técnicas de aprendizaje automática para
la detección de anomalías.

La producción de alertas es importante para dar a conocer sucesos importantes o
que requieran atención inmediata. Cuando ocurre un problema es deseable que se
notifique a través de un mensaje al responsable de solucionarlo.

La planificación de crecimiento consiste en estar preparado para cambios
futuros. Esto se puede lograr a partir del seguimiento de las métricas del
servidor y las aplicaciones. Es acerca de la habilidad de estudiar las
tendencias de los datos y usar ese conocimiento para tomar decisiones
informadas \cite[p.~15]{monitoreo:monitoring_with_grapfite}.

Tomaremos estas miradas como punto de partida para hacer un análisis más
profundo acerca de los aspectos más importantes a tener en cuenta a la hora de
diseñar un sistema de monitoreo eficaz.

