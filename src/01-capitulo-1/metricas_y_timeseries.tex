\subsection{Métricas y series de tiempo}
\label{metricas_y_timeseries}
Las métricas de monitoreo son colecciones de entradas de datos numéricos
organizadas en grupos de listas consecutivas y cronológicamente ordenadas. Cada
entrada de datos consiste en un valor de medición registrado, la marca de
tiempo en la que tuvo lugar la medición y un conjunto de propiedades que la
describen.

Cuando las entradas de datos de una métrica son segmentadas en intervalos de
tiempo fijos y convertidas en un estadístico mediante una transformación
matemática, pueden ser presentadas como series de tiempo e interpretadas en
gráficos bidimensionales.

El largo de los intervalos entre valores, también llamado granularidad, depende
de los tipos de mediciones y el tipo de información que se quiere extraer.
Intervalos comunes incluyen 1, 5, 15 y 60 minutos, pero también es posible
mostrar intervalos tan pequeños como un segundo y tan extensos como un día.

La ventaja más importante de usar datos de serie de tiempo para el monitoreo es
su propiedad de ilustrar con precisión el proceso de cambio en un contexto de
datos históricos. Son una herramienta indispensable para encontrar la respuesta
a la pregunta crítica: ¿Qué ha cambiado y cuándo?

Los datos de este tipo tienen dos dimensiones: una de datos propiamente dicha y
otra de tiempo. Esto significa que dos registros independientes compartirán
siempre la dimensión de tiempo. Es por esto que este tipo de registros son
útiles para destacar relaciones correlativas entre datos de muchas fuentes.

\subsubsection*{Estadísticos}
\label{estadisticos}

Las métricas de monitoreo almacenan las entradas de datos y describen sus
propiedades. El proceso de generar y graficar series de tiempo implica
recuperar un subconjunto de entradas de datos especificando un conjunto de
propiedades, dividiéndolas en intervalos de tiempo espaciados uniformemente, y
matemáticamente resumir entradas de datos en cada intervalo. Esto es realizado
con el uso de \glspl{term:estadistico}. Algunos de los estadísticos de uso más
común son:

\begin{itemize}
  \item La cantidad de entradas por intervalo.
  \item La suma de valores de todas las entradas.
  \item El valor promedio de todas las entradas.
  \item Los \glspl{term:percentil}\footnote{pueden tener valores de 0 a 100} de
    los valores entrada, incluyendo el mínimo\footnote{p0}, el
    máximo\footnote{p100} y la mediana\footnote{p50}.
  \item La \gls{term:desviacionestandar} del promedio en la distribución de
    entradas recolectadas.
\end{itemize}

Los \glspl{term:estadistico} describen conjuntos de entradas observadas por sus
centros (promedio o mediana), el total (suma o cantidad), y la distribución y
propagación (\glspl{term:percentil} y desviaciones). Pueden resumir grandes
conjuntos de datos de forma compacta y concisa. Convertir muchos números en uno
sólo causa la pérdida de información, pero el resumen suele ser lo
suficientemente preciso como para sacar conclusiones confiables.

\subsubsection*{Distribución de frecuencia y \glspl{term:percentil}}
\label{distribucion_de_frecuencia_y_percentiles}

La distribución de frecuencia es un resumen de un conjunto de datos que combina
elementos numéricos en grupos a través de un proceso conocido como
\eng{binning}. Esta distribución puede ser ilustrada como un \gls{term:histograma}, que
los presenta de manera que permite a las personas ver su tamaño relativo de
forma rápida.

Los \glspl{term:histograma} a menudo caen a lo largo de una distribución
normal, con sus columnas encajando justo debajo de la famosa curva gaussiana.
Pero los datos del sistema no suelen ser tan regulares y suelen mostrar una
cola larga del lado derecho; lo que significa que la distribución se encuentra
inclinada hacia el lado izquierdo.

Hay una razón simple para eso: el límite inferior en el rendimiento es un
límite duro: el mejor caso de tiempo de respuesta debe ser siempre más de cero
unidades de tiempo. Su límite superior es uno blando: teóricamente se puede
esperar por siempre.

Los \glspl{term:estadistico} graficados como valores en el tiempo son
convenientes para observar cambio, pero las series de tiempo no revelan
necesariamente la verdadera naturaleza de los datos. Es posible extraer datos
de entradas provenientes de análisis de logs sin conexión y presentarlos en un
\gls{term:histograma} para mostrar su frecuencia relativa. Esta información da
a los operadores una buena idea de qué valor esperar de una entrada típica y
cómo se ve la distribución de entradas detrás de cada valor resumido.

Una forma alternativa de resumir la distribución de frecuencias es un gráfico
de \glspl{term:percentil}. Para cualquier conjunto de entradas, un
\gls{term:percentil} es un número real en el rango de 0 a 100 con un valor
correspondiente de ese conjunto. El número en un \gls{term:percentil}
particular muestra cuántos valores son más pequeños que el valor de ese
\gls{term:percentil}.

Los \glspl{term:percentil} se calculan clasificando el conjunto de entradas por
valor en orden ascendente, encontrando el rango para un \gls{term:percentil}
dado (es decir, la dirección del valor en la lista ordenada) y buscando el
valor por rango.

El \gls{term:percentil} 0 es la medida del valor más bajo (el primer elemento
de una lista ordenada, o sea el mínimo), y el \gls{term:percentil} 100 es el
valor máximo registrado (el último elemento de la lista, o sea el máximo). Al
\gls{term:percentil} 50 se lo conoce comúnmente como la mediana, y representa
el valor medio en el conjunto

Los \glspl{term:percentil} facilitan la interpretación de la distribución; por
ejemplo, para medir el tiempo de respuesta el valor de \gls{term:percentil} 98
en 3 segundos significa que el 98\% de todas las solicitudes se completó en 3
segundos o menos. Por el contrario, el 2\% de las solicitudes más lentas
tomaron 3 segundos o más.

\subsubsection*{Tasa de cambio}
\label{tasa_de_cambio}

La tasa de cambio ilustra el grado de cambio entre valores en una serie de
tiempo u otra curva. Efectivamente, cuando la pendiente del gráfico original
está aumentando, su tasa de cambio tiene valores positivos. Por el contrario,
cuando la pendiente desciende, los valores de la serie de tasas de cambios son
negativos.

La tasa de cambio derivada de una serie de tiempos puede ser presentada como
otra serie de tiempo. La tasa de cambio es una herramienta conceptual útil para
ilustrar niveles de crecimiento o disminución con el tiempo. Es usada
extensivamente con métricas de conteo para expresar el número de incrementos
del contador por intervalo de tiempo.

\subsection*{Granularidad de tiempo}
\label{granularidad_de_tiempo}

Los datos en una función de serie de tiempos son presentados en una
granularidad de tiempo fijo. Una granularidad fina se traduce en períodos
cortos de valores. Cuanto más gruesa sea la granularidad, más largo será el
período.

Las métricas de granularidad fina tienden a revelar el momento exacto en que
ocurrió un evento y, por lo tanto, son útiles para describir líneas de tiempo y
encontrar dirección en las relaciones causales. Sin embargo podrían ser más
costosas de almacenar.

Las métricas de granularidad gruesa, por el otro lado, son mucho más adecuadas
para ilustrar tendencias.

La selección de la granularidad correcta para presentar una métrica es
importante para una interpretación precisa de los datos. Tanto las medidas
demasiado granulares como las demasiado gruesas pueden oscurecer el punto que
está tratando de transmitir.

Algunos sistemas de monitoreo usan intervalos constantes predefinidos y no
permiten al usuario modificar su granularidad, mientras que otros sistemas
permiten especificar períodos arbitrarios.

Sin embargo, siempre se requiere un intervalo mínimo. Incluso si fuéramos
capaces de presentar eventos en una escala de tiempo continuo (con una
granularidad infinitesimalmente fina), probablemente no sería muy útil.

En el mundo real no pueden ocurrir dos eventos exactamente al mismo tiempo, por
lo que el registro de eventos en una escala continua no podría hacer que el
número de eventos se acumulara en el gráfico.

Pasando al otro extremo, si el intervalo de datos es extremadamente largo, como
un año, la salida será un pequeño número de enormes colecciones de eventos. Eso
puede ser algo útil para los propósitos de análisis de datos pero no para
monitoreo.

\subsection*{Agregación de métricas}
\label{agregacion_de_metricas}

En sistemas distribuidos, las entradas de datos para la misma métrica provienen
de muchas fuentes. Por ejemplo, en un grupo de servidores \eng{web} detrás
de un balanceador de carga, reportando estadísticas acerca de solicitudes
siendo servidas, los datos de solicitud podrían ser vistos en la forma de
múltiples funciones de series de tiempo, una por cada servidor \eng{web},
graficadas unas contra otras. Sin embargo, la agregación podría combinar los
resultados de los servidores \eng{web} en una única secuencia de registros
de serie de tiempos con un total de todas las solicitudes.

La agregación permite obtener una vista general de los datos y simplifica el
gráfico, pero la capacidad de explorar para ver cada fuente de datos no es
menos importante. Si  uno de los servidores deja de tomar solicitudes,
reportará valores en cero, o directamente dejará de enviar entradas por
completo. La falla podría ser detectada mirando las métricas individuales del
servidor, pero no necesariamente se mostrarían en el gráfico agregado.

Muchas fallas, sin embargo, son mucho más sutiles y se manifiestan a través de
pequeñas depresiones en el número de peticiones en lugar de su desaparición
completa. En estos casos, es también deseable agregar valores de todas las
fuentes y presentarlos como una única métrica para mostrar el efecto
acumulativo \cite[p.~21-25]{monitoreo:efective_monitoring_and_alerting}.

