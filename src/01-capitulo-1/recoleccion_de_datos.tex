\subsection{Recolección de datos}
\label{recoleccion-de-datos}

El proceso de monitoreo comienza con la acumulación de datos por agentes de
recolección.

Un agente de recolección es un programa de software especializado que capta
información significativa sobre entidades monitoreadas como pueden ser
distintos tipos de terminales, bases de datos o dispositivos de red, etc., la
encapsula en entradas de datos cuantitativos y reporta estas entradas a un
sistema de monitoreo en intervalos regulares.

Estas entradas son transformadas en métricas que pueden ser almacenadas en
series temporales, es decir, secuencias de datos, observaciones o valores
ordenados cronológicamente.

La recolección de datos puede ser un proceso continuo o también puede darse de
forma periódica en intervalos regulares de tiempo, dependiendo la naturaleza de
las mediciones y del costo de los recursos involucrados en dicha recolección de
datos.

Según el conocimiento de la implementación de las entidades a monitorear, los
agentes de recolección pueden clasificarse en caja blanca o caja negra.

Los agentes de recolección de caja blanca son aquellos que trabajan sobre la
estructura interna de las entidades a monitorear. Entre ellos podemos
encontrar:

\begin{itemize}
  \item \textbf{Log Parsers:} los analizadores gramaticales o parsers de logs
    extraen información específica proveniente de las entradas de los logs,
    tales como códigos de estado y tiempos de respuesta de las peticiones desde
    los logs de los servidores \eng{web}.

  \item \textbf{Log Scanners:} los escaneadores de logs cuentan las ocurrencias
    de una cadena de caracteres en las líneas de los archivos de logs definidos
    por una expresión regular. Por ejemplo, para obtener información de errores
    regulares y críticos es posible calcular el número de ocurrencias de la
    expresión regular \texttt{/ERROR|CRITICAL/} dentro de un archivo de logs.

  \item \textbf{Interface readers:} los lectores de interfaces leen e
    interpretan interfaces de sistemas y dispositivos. Un ejemplo son los
    lectores de temperatura y humedad de dispositivos especializados, o el
    pseudo sistema de ficheros \texttt{/proc}, que permite acceso a información del
    kernel sobre los procesos en sistemas UNIX.
\end{itemize}

Los agentes de recolección de caja negra son aquellos que toman información sin
adentrarse en la implementación de dichas entidades. Entre ellos se encuentran:

\begin{itemize}
  \item \textbf{Probers:} son programas que se ejecutan fuera del sistema a
    monitorear enviando peticiones al sistema en cuestión para confirmar su
    respuesta. De esta forma trabajan las peticiones \gls{term:ping}. Otro
    ejemplo de este tipo de agentes son las solicitudes \gls{acro:http} hacia
    un sitio \eng{web} con el propósito de verificar su disponibilidad.

  \item \textbf{Sniffers:} son programas informáticos que registran la
    información que envían los periféricos. Por ejemplo, pueden monitorear
    interfaces de red y analizar estadísticas de tráfico como el número de
    paquetes transmitidos bajo un protocolo.
    \cite[p.~15-16]{monitoreo:efective_monitoring_and_alerting}
\end{itemize}

En el marco de una infraestructura de software, los datos a recolectar por un
agente pueden ser obtenidos de varias fuentes. Por ejemplo, de una computadora
es posible obtener información del uso de CPU o espacio de disco utilizado,
utilización de memoria. A partir del tráfico \eng{web} es posible medir el
uso del ancho de banda por usuario, aplicación, protocolo y grupo de
direcciones \gls{acro:ip}. Se puede medir el rendimiento y uso de una
aplicación \eng{web}, y también reglas de negocio, como por ejemplo
cantidad de hitos alcanzados.

Estas métricas pueden ser útiles para usuarios con diferentes necesidades. Por
ejemplo, un sistema de monitoreo podría ser usado por un gerente, un líder de
proyecto, un desarrollador, un profesional de \gls{acro:it}, un analista de
datos y un cliente.

Teniendo en cuenta el rol que cumple una persona en un proyecto, podemos
averiguar qué información puede serle útil y a partir de allí elegir el mejor
método para almacenar esta información, analizar cómo construirla a partir de
los datos y localizar dónde y cómo obtener dichos datos.

Al profesional de \gls{acro:it} le podría servir tener un tablero donde se
observe la salud de los sistemas en tiempo real. Además podría interesarle
conocer la disponibilidad de los servicios, la cantidad de tráfico en la red,
el uso de \gls{acro:cpu} y memoria, el uso de memoria \gls{term:swap} y la
cantidad de disco utilizado.

Un líder de proyecto puede querer disponer de gráficos donde se visualice la
cantidad de despliegues por semana, el número de problemas resueltos por los
programadores, el rendimiento de la aplicación y el tiempo promedio de
respuesta de la aplicación.

Un desarrollador puede encontrar utilidad en la visibilización del rendimiento
de diferentes funcionalidades de la aplicación, la eficiencia en tiempo de los
accesos a las bases de datos y la cantidad de accesos a la aplicación en
distintos períodos de tiempo. Además puede interesarle contar con información
del comportamiento de los recursos del servidor en el ambiente productivo.
Podemos decir que al desarrollador le interesan principalmente las métricas de
aplicación.

Las métricas de negocio están por sobre las métricas de la aplicación aunque
tienen un vínculo bastante estrecho. Por ejemplo, si se piensa medir el tiempo
de ejecución de una transacción de pago como métrica de aplicación, podrían
realizarse métricas de negocio con el contenido de estas solicitudes: número de
nuevos usuarios/clientes, número de ventas, ventas por valor o ubicación, o
cualquier otra cosa que ayude a medir el estado de un negocio. Estas métricas
están compuestas del tipo de información que pueden interesarle al propietario
del sistema \cite[p.~446]{monitoreo:art_of_monitoring}. Hay que tener en
cuenta, como dice autor en The Art Of Monitoring, que “El primer cliente de su
sistema de monitoreo es el negocio. El monitoreo existe para apoyar el negocio
y para asegurarse de que continúa haciendo su trabajo.”
\cite[p.~8]{monitoreo:art_of_monitoring}.

