\newpage
% Glosario
\newglossaryentry{term:ejemplo} {
  name = {nombre del ejemplo},
  description = {descipcion del ejemplo},
  sort = {nombredelejemplo}
}

\newglossaryentry{term:estadistico} {
  name        = {estadístico},
  description =
    {medida cuantitativa, derivada de un conjunto de datos de una muestra, con
    el objetivo de estimar o inferir características de una población o modelo
    estadístico},
  sort        = {estadistico}
}

\newglossaryentry{term:percentil} {
  name        = {percentil},
  plural      = {percentiles},
  description =
    {medida de posición usada en estadística que indica, una vez ordenados los
    datos de menor a mayor, el valor de la variable por debajo del cual se
    encuentra un porcentaje dado de observaciones en un grupo de observaciones},
  sort        = {percentil}
}

\newglossaryentry{term:desviacionestandar} {
  name        = {desviación estándar},
  description =
    {es una medida de dispersión para variables de razón y de intervalo. Se
    define como la raíz cuadrada de la varianza de la variable},
  sort        = {desviacionestandar}
}

\newglossaryentry{term:histograma} {
  name        = {histograma},
  description =
    {representación gráfica de una distribución de frecuencias por medio de
    rectángulos, cuyas anchuras representan intervalos de la clasificación y
    cuyas alturas representan las correspondientes frecuencia.
    \url{http://dle.rae.es/?id=KWdW8dL}},
  sort        = {histograma}
}

\newglossaryentry{term:ping} {
  name        = {ping},
  description =
    {utilidad diagnóstica1 en redes de computadoras que comprueba el estado de
    la comunicación del host local con uno o varios equipos remotos de una red
    IP},
  sort        = {ping}
}

\newglossaryentry{term:swap} {
  name        = {swap},
  description =
    {zona del disco (un fichero o partición) que se usa para guardar las
    imágenes de los procesos que no han de mantenerse en memoria física},
  sort        = {swap}
}

\newglossaryentry{term:host} {
  name        = {host},
  description =
    {computadora conectada a una red, que proveen y utiliza servicios de ella},
  sort        = {host}
}

\newglossaryentry{term:its} {
  name        = {sistema de seguimiento de incidentes},
  description =
    {paquete de software que administra y mantiene listas de incidentes,
    conforme son requeridos por una institución},
  sort        = {its}
}

\newglossaryentry{term:linux} {
  name        = {Linux},
  description =
    {término empleado para referirse a la combinación del sistema operativo
    GNU, y el núcleo (\eng{kernel}) Linux},
  sort        = {linux}
}

\newglossaryentry{term:php} {
  name        = {PHP},
  description =
    {\eng{Hypertext Preprocessor}
    \url{http://php.net/manual/es/intro-whatis.php}},
  sort        = {php}
}

\newglossaryentry{term:ruby} {
  name        = {Ruby},
  description = {\url{https://www.ruby-lang.org/es/}},
  sort        = {ruby}
}

\newglossaryentry{term:devops} {
  name        = {DevOps},
  description = 
    {\eng{develpment-operations}. Se trata de una cultura o movimiento que se
    centra en la comunicación, colaboración e integración entre desarrolladores
    de software y los profesionales en las tecnologías de la información
    (IT)},
  sort        = {devops}
}

\newglossaryentry{term:lean} {
  name        = {Lean},
  description = {Buscar un buen enlace},
  sort        = {lean}
}

\newglossaryentry{term:delivery_continuo} {
  type        = \acronymtype,
  name        = {CD},
  description = {\eng{Continuous delivery}. Buscar un buen enlace}, 
  first       = {Entrega Continua (CD)},
  sort        = {delivery_continuo}
}

\newglossaryentry{term:ssh} {
  type        = \acronymtype,
  name        = {SSH},
  description = 
    {\eng{Secure Shell}. Programa que permite acceder a máquinas remotas a
    través de una red}
  first       = {Secure Shell (SSH)},
  sort        = {ssh}
}

\newglossaryentry{term:htop} {
  name        = {htop},
  description =
    {programa para visualizar procesos de forma interactiva.
    \url{http://hisham.hm/htop/}}
  sort        = {htop}
}

\newglossaryentry{term:ror} {
  name        = {Rails},
  type        = \acronymtype,
  description =
    {framework Ruby enfocado al desarrollo de aplicaciones web.
    \url{https://rubyonrails.org/}}
  first       = {Ruby on Rails (Rails)},
  sort        = {ror}
}

\newglossaryentry{term:mysql} {
  name        = {MySQL},
  description =
    {sistema de gestión de bases de datos relacional.
    \url{https://www.mysql.com/}}
  sort        = {mysql}
}

\newglossaryentry{term:gema} {
  name        = {gema},
  description =
    {las gemas son plugins y/o códigos añadidos a nuestros proyectos Ruby on
    Rails, que nos permiten extenderlo con nuevas funcionalidades de manera
    modular},
  sort        = {gema}
}

\newglossaryentry{term:stack_trace} {
  name        = {\eng{stake trace}},
  description = {pila de llamadas},
  sort        = {stack_trace}
}

\newglossaryentry{term:contenedor} {
  name        = {\eng{stake trace}},
  plural      = {contenedores},
  description = {tecnología de virtualización en el nivel de sistema operativo},
  sort        = {contenedor}
}

\newglossaryentry{term:docker} {
  name        = {Docker},
  description = {plataforma de contenedores de software},
  sort        = {docker}
}

% Acronimos
\newacronym{acro:ej}{EJ}{\eng{EXAMPLE}}

\newacronym{acro:cespi}{CeSPI}{Centro Superior para el Procesamiento de la
Información}
\newacronym{acro:unlp}{UNLP}{Universidad Nacional de La Plata}
\newacronym{acro:http}{HTTP}{\eng{Hypertext Transfer Protocol}}
\newacronym{acro:ip}{IP}{\eng{Internet Protocol}}
\newacronym{acro:web}{web}{\eng{World Wide Web}}
\newacronym{acro:it}{IT}{\eng{Information Technology}}
\newacronym{acro:cpu}{CPU}{\eng{Central Processing Unit}}
\newacronym{acro:snmp}{SNMP}{\eng{Simple Network Management Protocol}}
\newacronym{acro:xml}{XML}{\eng{Extensible Markup Language}}
\newacronym{acro:json}{JSON}{\eng{JavaScript Object Notation}}
