\newpage
% Glosario
\newglossaryentry{term:ejemplo} {
  name = {nombre del ejemplo},
  description = {descipcion del ejemplo},
  sort = {nombredelejemplo}
}

\newglossaryentry{term:estadistico} {
  name        = {estadístico},
  description =
    {medida cuantitativa, derivada de un conjunto de datos de una muestra, con
    el objetivo de estimar o inferir características de una población o modelo
    estadístico},
  sort        = {estadistico}
}

\newglossaryentry{term:percentil} {
  name        = {percentil},
  plural      = {percentiles},
  description =
    {medida de posición usada en estadística que indica, una vez ordenados los
    datos de menor a mayor, el valor de la variable por debajo del cual se
    encuentra un porcentaje dado de observaciones en un grupo de observaciones},
  sort        = {percentil}
}

\newglossaryentry{term:desviacionestandar} {
  name        = {desviación estándar},
  description =
    {es una medida de dispersión para variables de razón y de intervalo. Se
    define como la raíz cuadrada de la varianza de la variable},
  sort        = {desviacionestandar}
}

\newglossaryentry{term:histograma} {
  name        = {histograma},
  description =
    {representación gráfica de una distribución de frecuencias por medio de
    rectángulos, cuyas anchuras representan intervalos de la clasificación y
    cuyas alturas representan las correspondientes frecuencia.
    \url{http://dle.rae.es/?id=KWdW8dL}},
  sort        = {histograma}
}

\newglossaryentry{term:ping} {
  name        = {ping},
  description =
    {utilidad diagnóstica1 en redes de computadoras que comprueba el estado de
    la comunicación del host local con uno o varios equipos remotos de una red
    IP},
  sort        = {ping}
}

\newglossaryentry{term:swap} {
  name        = {swap},
  description =
    {zona del disco (un fichero o partición) que se usa para guardar las
    imágenes de los procesos que no han de mantenerse en memoria física},
  sort        = {swap}
}

\newglossaryentry{term:host} {
  name        = {host},
  description =
    {computadora conectada a una red, que proveen y utiliza servicios de ella},
  sort        = {host}
}

\newglossaryentry{term:its} {
  name        = {sistema de seguimiento de incidentes},
  description =
    {paquete de software que administra y mantiene listas de incidentes,
    conforme son requeridos por una institución},
  sort        = {its}
}

\newglossaryentry{term:linux} {
  name        = {Linux},
  description =
    {término empleado para referirse a la combinación del sistema operativo
    GNU, y el núcleo (\eng{kernel}) Linux},
  sort        = {linux}
}

% Acronimos
\newacronym{acro:ej}{EJ}{\eng{EXAMPLE}}

\newacronym{acro:cespi}{CeSPI}{Centro Superior para el Procesamiento de la
Información}
\newacronym{acro:unlp}{UNLP}{Universidad Nacional de La Plata}
\newacronym{acro:devops}{DevOps}{\eng{develpment-operations}}
\newacronym{acro:http}{HTTP}{\eng{Hypertext Transfer Protocol}}
\newacronym{acro:ip}{IP}{\eng{Internet Protocol}}
\newacronym{acro:web}{web}{\eng{World Wide Web}}
\newacronym{acro:it}{IT}{\eng{Information Technology}}
\newacronym{acro:cpu}{CPU}{\eng{Central Processing Unit}}
\newacronym{acro:snmp}{SNMP}{\eng{Simple Network Management Protocol}}
\newacronym{acro:xml}{XML}{\eng{Extensible Markup Language}}
\newacronym{acro:json}{JSON}{\eng{JavaScript Object Notation}}

