\section{Tecnologías del CeSPI}
\label{anexo:A}

En esta sección describiremos aspectos de algunas herramientas usadas por el
\gls{acro:cespi}, que si bien no es necesario conocerlos a fondo para entender
la tesis, han formado parte de nuestra investigación y creemos que es
importante que las mencionemos.

\subsection{Ruby}

Ruby es un lenguaje de programación interpretado y orientado a objetos, creado
en 1995 por Yukihiro Matsumoto. Ruby combina una sintaxis inspirada en Python y
Perl, con características de programación orientada a objetos similares a
Smalltalk. Su implementación oficial es distribuida bajo una licencia de
software libre.

En Ruby, todos los tipos de datos son un objeto, incluidas las clases y tipos
que otros lenguajes definen como primitivas. Toda función es un método y las
variables siempre son referencias a objetos, y no los objetos mismos.

Ruby soporta herencia con enlace dinámico, mixins y métodos definidos por
instancia. Ruby no soporta herencia múltiple, pero esta se puede imitar
haciendo que una clase importe módulos como si fueran mixins.

Ruby es un lenguaje multiparadigma, en el sentido en que permite programación
procedural, con orientación a objetos y funcional. Todas las sentencias tienen
valores, y las funciones devuelven la última evaluación. Ruby soporta
introspección, reflexión y metaprogramación~\cite{ruby}.

\subsection{Ruby on Rails}

Ruby on Rails, también conocido como RoR o Rails, es un framework de
aplicaciones \eng{web} de código abierto escrito en el lenguaje de programación Ruby,
siguiendo el patrón Modelo Vista Controlador (MVC).

Trata de combinar la simplicidad con la posibilidad de desarrollar aplicaciones
del mundo real escribiendo menos código que con otros frameworks y con un
mínimo de configuración.

El lenguaje de programación Ruby permite la metaprogramación, de la cual Rails
hace uso, lo que resulta en una sintaxis que muchos de sus usuarios encuentran
muy legible. Rails se distribuye a través de RubyGems, que es el formato
oficial de paquete y canal de distribución de bibliotecas y aplicaciones
Ruby~\cite{ror}.

\subsection{Docker}

Docker es un proyecto de código abierto que permite la automatización del
despliegue de aplicaciones dentro de \glspl{term:contenedor} de software en Linux. Los
contenedores proporcionan una capa adicional de abstracción y automatización de
virtualización a nivel de sistema operativo.

Docker utiliza características de aislamiento de recursos del kernel de Linux,
tales como cgroups y espacios de nombres para permitir que contenedores
independientes se ejecuten dentro de una sola instancia de Linux, evitando la
sobrecarga de iniciar y mantener máquinas virtuales.

Docker es una herramienta que puede empaquetar una aplicación y sus
dependencias en un \gls{term:contenedor} virtual que se puede ejecutar en cualquier
servidor Linux. Esto ayuda a permitir la flexibilidad y portabilidad en donde
la aplicación se puede ejecutar, por ejemplo en instalaciones físicas, la nube
pública o nubes privadas.

Docker implementa una interfaz para proporcionar \glspl{term:contenedor} livianos que
ejecutan procesos de forma aislada.

A diferencia de una máquina virtual, un \gls{term:contenedor} \gls{term:docker} no requiere incluir
un sistema operativo independiente. En su lugar se basa en las funcionalidades
del kernel y utiliza el aislamiento de recursos (CPU, memoria, bloque de
entrada y salida y la red, entre otros) y espacios de nombres separados para
aislar la aplicación del sistema operativo.

A partir del uso de contenedores, los recursos pueden ser aislados y los
servicios restringidos. Se otorga a los procesos la capacidad de tener una
visión casi completamente privada del sistema operativo, con su propio
identificador de espacio de proceso, la estructura del sistema de archivos y
las interfaces de red. Múltiples \glspl{term:contenedor} comparten el mismo núcleo, pero
cada \gls{term:contenedor} puede ser restringido a usar sólo una cantidad definida de
recursos.

En la práctica, \gls{term:docker} puede ser utilizado para simplificar la creación de
sistemas altamente distribuidos. El despliegue de nodos puede realizarse a
medida que se disponga de recursos o cuando se necesiten más nodos, lo que
permite la construcción de una plataforma como servicio. \gls{term:docker} también
facilita el armado y funcionamiento de tareas de carga de
trabajo~\cite{docker}.

\subsection{Docker Compose}
\label{anexo_compose}

Docker Compose es una herramienta para definir y ejecutar aplicaciones Docker
multicontenedor. Con Compose, es posible definir la configuración de los
servicios de una aplicación en un archivo, y luego crear e iniciar todos los
servicios desde esa configuración con un único comando.

Compose ha demostrado ser una excelente herramienta para desarrollar y probar
ambientes de desarrollo, y para definir flujos de integración
continua~\cite{compose}. Para usar Compose, se necesitan seguir los siguientes
pasos:

Definir el ambiente de una aplicación con un Dockerfile
\footnote{Cf. \url{https://docs.docker.com/engine/reference/builder/}}, para
poder reproducirlo en cualquier lugar.  Definir los servicios que constituyen
la aplicación en un archivo llamado docker-compose.yml
\footnote{Cf.  \url{https://docs.docker.com/compose/compose-file/}}, de forma
que puedan ser ejecutados juntos en un ambiente aislado.  Finalmente, ejecutar
el comando docker-compose up y Compose iniciara y correrá la aplicación en su
totalidad\footnote{Cf. \url{https://docs.docker.com/compose/reference/up/}}.


\subsection{Rancher}

Rancher es una plataforma de código abierto para manejar \glspl{term:contenedor} que
facilita la tarea de correr cualquier aplicación basada en \glspl{term:contenedor} en
cualquier infraestructura. Soporta Kubernetes, Mesos y Docker Swarn.

Fue diseñado para resolver todos los desafíos críticos necesarios para correr
aplicaciones en contenedores. Rancher provee un conjunto completo de servicios
de infraestructura para contenedores, incluyendo servicios de red, servicios de
almacenamiento, manejo de hosts y balanceo de carga. Todos estos servicios
funcionan sobre cualquier infraestructura y facilitan la tarea de desplegar y
manejar aplicaciones de forma confiable~\cite{rancher}.

Rancher consiste en cuatro componentes principales:

\begin{itemize}

  \item \textbf{Orquestación de infraestructura:}
    Rancher toma recursos informáticos de cualquier nube pública o privada en
    la forma de hosts de Linux. Cada \gls{term:host} de \gls{term:linux} puede ser una máquina
    virtual o física. Rancher no espera más de cada \gls{term:host} que CPU, memoria,
    almacenamiento en el disco local y conectividad en la red.

    Rancher implementa una capa protable de servicios de infraestructura
    diseñados específicamente para alimentar aplicaciones en contenedores. Los
    servicios de infraestructura de rancher incluyen servicios de red,
    almacenamiento, balance de carga, DNS y seguridad. Los servicios de
    infraestructura de Rancher son típicamente desplegados como contenedores, y
    de esta forma el mismo servicio de infraestructura de Rancher puede correr
    en cualquier \gls{term:host} de linux de cualquier nube.

  \item \textbf{Orquestación y programación de contenedores:}
    Muchos usuarios eligen correr aplicaciones en \glspl{term:contenedor} usando un
    framework de orquestación y programación de contenedores. Rancher incluye
    una distribución de todos los frameworks populares hoy en día, incluyendo
    Docker Swarm, Kubernetes y Mesos. Un mismo usuario puede crear múltiples
    clusters de Swarm o Kubernetes, y luego usar las herramientas nativas de
    estos frameworks para manejar sus aplicaciones.

    Además de Swarm, Kubernetes, y Mesos, Rancher soporta su propio framework
    de orquestación y programación de contenedores, llamado Cattle. Cattle fue
    originalmente diseñado como una extensión de Docker Swarm, pero fue
    divergiendo a medida que continuó el desarrollo de ambos.

  \item \textbf{Catálogo de aplicación:}
    Un usuario de Rancher puede desplegar una aplicación multicontenedor en
    clusters desde el catálogo de aplicación presionando un único botón. Los
    usuario pueden manejar las aplicaciones desplegadas y ejecutar
    actualizaciones completamente automatizadas cuando nuevas versiones de la
    aplicación se vuelven disponibles. Rancher mantiene un catálogo público que
    consiste en aplicaciones populares creadas por la comunidad de Rancher. Un
    usuario puede crear su propio catálogo privado.

  \item \textbf{Control de nivel empresarial:}
    Rancher soporta plugins de autenticación de usuarios flexibles y viene por
    defecto con integración de autenticación de usuarios con ActiveDirectory,
    LDAP y Github. Rancher soporta Control de Acceso basado en Roles (RBAC) a
    nivel de ambientes, permitiendo a los usuarios y grupos compartir o negar
    acceso a, por ejemplo, ambientes de desarrollo y producción.

\end{itemize}

\clearpage
