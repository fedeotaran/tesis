\subsection{Elección de la herramienta y funcionamiento}
\label{eleccion-herramienta}

Hemos elegido la herramienta \gls{term:kapacitor} para implementar las alertas
en nuestra solución. Las razón detrás de nuestra decisión es que
\gls{term:kapacitor} cuenta con una \gls{acro:dsl} realmente simple, la cual
nos ha permitido implementar algunas alertas complejas de forma verdaderamente
rápida. Además, el hecho de que haya sido desarrollado por la misma empresa
que creó \gls{term:influx} le ha otorgado una integración excelente con el
mismo.

\gls{term:kapacitor} es un framework de procesamiento de datos de código
abierto, que permite crear alertas, correr procesos ETL (de extracción,
transformación y carga) y detectar anomalías. \gls{term:kapacitor} es parte del
\eng{stack} TICK, junto con Telegraf, \gls{term:influx} y Chronograf.

\gls{term:kapacitor} permite procesar datos por streaming y por lotes. También
consultar los datos de \gls{term:influx} en un programa y recibir datos a
través del protocolo de línea y cualquier otro método que \gls{term:influx}
admita.

Adicionalmente, admite realizar cualquier transformación actualmente posible en
el lenguaje de consulta InfluxQL y almacenar los datos transformados en
\gls{term:influx}. Además permite añadir funciones personalizadas definidas por
el usuario para detectar anomalías.

\gls{term:kapacitor} puede ser configurado para notificar alertas a distintos
destinos.  Por ejemplo se pueden enviar a un archivo de logs, una URL con
método \gls{acro:http} POST, a través de correo electrónico y servicios como
HipChat, \gls{term:alerta}, Sensu, Slack y PagerDuty, entre otros. Incluso es
posible ejecutar un comando dirigiendo los datos de la alerta por
\gls{acro:stdin}\footnote{\url{https://docs.influxdata.com/kapacitor/v1.2/}}.

\gls{term:kapacitor} cuenta con una \gls{acro:dsl} llamada TICKscript para
generar alertas. Esta \gls{acro:dsl} es usada para definir canales de
procesamiento de datos en \gls{term:kapacitor}.

\gls{term:kapacitor} usa este lenguaje para definir canales de procesamiento de
datos, o pipelines. Un pipeline es un conjunto de nodos que procesa datos y
aristas que conectan nodos. Los pipelines en \gls{term:kapacitor} son grafos
acíclicos dirigidos, lo que significa que cada arista tiene una dirección en la
que fluyen los datos y no puede haber ningún ciclo en el pipeline.

\gls{term:kapacitor} puede ser descargado e instalado en Debian o Ubuntu con
los siguientes comandos:

\begin{lstlisting}

wget https://dl.influxdata.com/kapacitor/releases/kapacitor_1.2.0_amd64.deb
sudo dpkg -i kapacitor_1.2.0_amd64.deb

\end{lstlisting}

Para correr el servicio de \gls{term:kapacitor} simplemente es necesario
ejecutar el siguiente comando en la terminar:

\begin{lstlisting}

sudo service kapacitor start

\end{lstlisting}

Además, es posible usarlo como \gls{term:contenedor} de \gls{term:docker}, a través de la
imágen oficial de \gls{term:kapacitor}
\footnote{\url{https://hub.docker.com/_/kapacitor/}}, con el comando:

\begin{lstlisting}

docker pull kapacitor

\end{lstlisting}
