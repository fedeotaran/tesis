Cómo mencionamos en la sección 1.6, las alertas nos brindan la posibilidad de ser notificados de forma automática sobre eventos que nos parecen importantes. Si no se cuenta con un sistema de alertas, la mejor forma de identificar un problema es visualizar el funcionamiento de las aplicaciones, de forma constante y manual.

Esto puede ser muy costoso, y es uno de los motivos por los que la elección de una herramienta de generación de alertas que nos permita describir las condiciones que deben complirse para ser notificados de situaciones importantes se torna fundamental.

En este capítulo describiremos las herramientas que nos han parecido más adecuadas para resolver la importante tarea de las alertas, al mismo tiempo que daremos las razones de estas elecciones. Además mostraremos cómo configurar estas herramientas para ser notificados de eventos importantes que ocurran en la infraestructura.

En la sección 6.1 describiremos la herramienta Kapacitor, que permite la definición de alertas y el envío de las mismas a distintas fuentes, y explicaremos cómo instalarla y configurarla para generar importantes alertas.

En la sección 6.2 demostraremos cómo utilizar el lenguaje TICKscript para definir alertas. Primero con un ejemplo sencillo, y luego con un ejemplo más complejo.

En la sección 6.3 explicaremos los problemas encontrados en la generación de múltiples alertas y cómo lo hemos solucionado utilizando la herramienta de nombre Alerta.
