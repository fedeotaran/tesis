\subsection{Objetivo}
\label{objetivo}

El objetivo que tiene la investigación de esta tesis es plasmar el resultado de
comparar y proponer una estrategia de recolección, análisis y utilización de
información de la infraestructura y las aplicaciones del \gls{acro:cespi}, que
permita simplificar y enriquecer su análisis posterior y resulte en un
monitoreo eficiente.

Se espera que cualquier problema con las aplicaciones pueda ser detectado y
analizado en un contexto global, y que diferentes fuentes de registros de datos
puedan ser correlacionadas para aprender aún más sobre el estado de los
proyectos. Para lograr esto realizará una evaluación el estado de arte de las
estrategias de recolección de datos, estadísticas, monitoreo y generación de
alertas.  A partir de este análisis, habrá que relevar las herramientas que
permitan implementar dichas estrategias, permitiéndo así proceder hacia un
diseño de una infraestructura que las utilice.

Finalmente se integrará el sistema de monitoreo a la infraestructura actual de
forma automatizada mediante prácticas \gls{term:devops}, y buscará sentar las
bases para que se puedan obtener tableros de control que permitan visualizar el
estado de la infraestructura completa y que permita profundizar en resultados
según los criterios de los diferentes perfiles de usuario.

Se busca entonces con este procedimiento identificar los datos de importancia a
ser recolectados para alimentar un sistema de monitoreo que permita alertar de
forma inteligente cuando alguna parte crítica del sistema se vuelva inestable o
no responda a los resultados deseables. A su vez, se quiere conseguir la
disponibilidad en todo momento de una vista de alto nivel del funcionamiento de
toda la infraestructura, que facilite la toma de decisiones en distintos
niveles según los intereses de los usuarios.

Teniendo en cuenta todo lo mencionado hasta ahora, se buscará alcanzar una
solución de monitoreo que permita entender a la infraestructura no como piezas
separadas, sino como un conjunto de componentes correlacionados.
