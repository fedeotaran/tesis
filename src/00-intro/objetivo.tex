\subsection{Objetivo}
\label{objetivo}

El objetivo que buscamos al escribir esta tesis es plasmar el resultado de
investigar, comparar y proponer una estrategia de recolección, análisis y
utilización de información de la infraestructura y las aplicaciones del CeSPI,
que permita simplificar y enriquecer su análisis posterior y resulte en un
monitoreo eficiente.

Esperamos que cualquier problema con las aplicaciones pueda ser detectado y
analizado en un contexto global, y que diferentes fuentes de registros de datos
puedan ser correlacionadas para aprender aún más sobre el estado de los
proyectos.  Para lograr esto evaluaremos el estado de arte de las estrategias
de recolección de datos, estadísticas, monitoreo, generación de alertas y
respuestas automáticas. A partir de este análisis, procederemos a relevar las
herramientas que permitan implementar dichas estrategias, permitiéndonos así
proceder hacia un diseño de una infraestructura que las utilice.

Finalmente se integrará el sistema de monitoreo a la infraestructura actual de
forma automatizada mediante prácticas DevOps, y se obtendrán tableros de
control que permitan visualizar el estado de la infraestructura completa en una
única pantalla, que permita profundizar en resultados según los criterios de
los diferentes perfiles de usuario.

Confiamos en que lograremos identificar los datos de importancia a ser
recolectados para alimentar un sistema de monitoreo que permita alertar de
forma inteligente cuando alguna parte crítica del sistema se vuelva inestable o
no responda a los resultados deseables. A su vez, esperamos conseguir la
disponibilidad en todo momento de una vista de alto nivel del funcionamiento de
toda la infraestructura, que facilite la toma de decisiones en distintos
niveles según los intereses de los usuarios.

Considerando todo lo mencionado hasta ahora, buscaremos alcanzar una solución
de monitoreo que permita entender a la infraestructura no como piezas
separadas, sino como un conjunto de componentes correlacionados.

