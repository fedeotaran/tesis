Los desarrolladores de sistemas informáticos de la Dirección de Desarrollo del
\gls{acro:cespi}, \gls{acro:unlp}, construyen aplicaciones para clientes
internos y externos a la \gls{acro:unlp}. En esta oficina se aplican técnicas
\gls{term:devops} para agilizar el despliegue y mantener en sintonía los
ambientes de desarrollo, testing y producción. Su equipo de trabajo está
integrado por más de una docena de profesionales en informática, los cuales
están a cargo de diferentes proyectos.

Parte importante de la implementación y mantenimiento de estos sistemas es la
recolección y análisis de datos de su funcionamiento. Este análisis es
importante para conocer la salud de los servidores y el rendimiento de los
procesadores, detectar fallas en el \eng{hardware} y \eng{software}, descubrir
comportamientos que afecten la funcionalidad de las aplicaciones y tomar
decisiones que mejoren la productividad y calidad de los servicios brindados.

Recoger datos y calcular estadísticas sobre servidores, aplicaciones y tráfico
puede aumentar la seguridad en la toma de decisiones y permitir la anticipación
a problemas.

La disciplina de recolectar datos de forma constante a lo largo del tiempo
facilita la comprensión de los eventos que ocurren en cada instante,
permitiendo contrastar un valor actual con valores previos. Esto proporciona la
capacidad de respaldar decisiones con datos reales.

En la actualidad existen una gran variedad de herramientas que pueden ser
usadas para tener un seguimiento de métricas sobre servicios de \eng{software}, y es
posible integrar estas herramientas para crear un sistema que permita obtener,
almacenar y visualizar los resultados. Hoy en día, no todos los proyectos
dentro del \gls{acro:cespi} cuentan con herramientas adecuadas para esta tarea.
