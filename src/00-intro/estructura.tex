\subsection{Estructura del documento}
\label{estructura}
El trabajo está compuesto por 10 capítulos. En el capítulo 1 establecemos el marco teórico de la tesis, que tomamos como eje de referencia y brinda el respaldo teórico. En el capítulo 2 explicamos el caso de estudio que vamos a abordar. De los capítulos 3 a 6 detallamos la solución propuesta y en los capítulos siguientes se encuentran las conclusiones, los trabajos futuros y los anexos respectivamentes.

A continuación haremos un breve resumen de los capítulos principales:

En el capítulo siguiente vamos a dar una introducción sobre el significado de monitoreo, explicando dónde radica su importancia, qué información le sirve a los distintos perfiles de usuarios y cómo diseñar un sistema de monitoreo considerando esa información.

En el capítulo 3 explicaremos el caso de estudio al que aplicaremos nuestra implementación. La metodología de trabajo del laboratorio, las tecnologías que forman parte de la infraestructura de la aplicación, el estado actual de la herramienta de monitoreo y nuestros objetivos a la hora de implementar la solución.

En el capítulo 4 comenzaremos explicando cómo realizaremos la obtención de datos, particularmente de los logs de las aplicaciones. Explicaremos qué son los logs, qué herramientas utilizamos para obtener información, y qué otro tipo de datos se pueden medir y recolectar.

En el capítulo 5 explicaremos cómo filtrar y almacenar datos obtenido de la aplicación en tiempo de ejecución para poder utilizarlos más adelante. Abarca la obtención de datos como: datos de red, datos de los recursos de los host, datos de los procesos, etc. Además, nombraremos las distintas tecnologías que se pueden aplicar para diferentes problemáticas.

En el capítulo 6 explicaremos para qué sirve la visualización de datos, cuestiones importantes a tener en cuenta, la ventaja que da la visualización para el entendimiento de los datos, herramientas utilizadas y los resultados obtenidos.

En el capítulo 7 explicaremos qué son las alertas, las dificultades de implementar un buen sistema de alertas y las herramientas utilizadas junto con los resultados obtenidos.

Finalmente en el capítulo 8 daremos a conocer las conclusiones de la experiencia obtenida a lo largo del desarrollo del trabajo.
