\subsection{Estructura del documento}
\label{estructura}

En el \autoref{cap1} se muestran y comparan distintas definiciones de
monitoreo, explicando dónde radica su importancia, qué información podría ser
útil a los distintos perfiles de usuarios y cómo diseñar un sistema de
monitoreo considerando esa información. Además, se explican algunos conceptos
importantes que están íntegramente relacionados con este tema.

En el \autoref{cap2} se expone el caso de estudio en el que se centra la
implementación sugerida. La metodología de trabajo del \gls{acro:cespi}, las
tecnologías que forman parte de la infraestructura de las aplicaciones, el
estado actual de la herramienta de monitoreo y los objetivos a la hora de
implementar la solución son los temas principales en este capítulo.

En el \autoref{cap3} se explica cómo filtrar y almacenar datos obtenidos en
tiempo real con el objetivo de poder utilizarlos más adelante. Se aborda la
recolección de datos de red, de recursos de los \gls{term:host}, de los
procesos y de las aplicaciones, entre otros. Además, se va a describir las
distintas tecnologías y herramientas usadas.

En el \autoref{cap4} se plantea cómo implementar la obtención de datos de los
\eng{logs} de las aplicaciones y de los servicios. Describir los problemas
comunes en la recolección de datos de \eng{logs} y las herramientas utilizadas
para resolverlos es el objetivo específico para este capítulo.

En el \autoref{cap5} se explica para qué sirve la visualización de datos,
cuestiones importantes a tener en cuenta, la ventaja que da la visualización
para el entendimiento de los datos, herramientas utilizadas y los resultados
obtenidos.

En el \autoref{cap6} se describe qué son las alertas, las dificultades de
implementar un buen sistema de alertas y las herramientas utilizadas junto con
los resultados obtenidos.

Finalmente se cuentan las conclusiones de la experiencia que hemos obtenido a
lo largo del desarrollo del trabajo y de nombran posibles trabajos futuros
que se pueden realizar tomando como base esta investigación.

En el \autoref{anexo:A} se desarrollan aspectos de algunas herramientas que
utilizan en el \gls{acro:cespi}.

En el \autoref{anexo:B} se mencionan otras herramientas que si bien no forman
parte de nuestra solución final, han formado parte de la investigación y que
pueden ser útiles para soluciones similares.

En el \autoref{anexo:C} se amplía el código utilizado en una parte de nuestra
solución.
