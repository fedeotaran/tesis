\subsection{Estructura del documento}
\label{estructura}

En el \autoref{cap1} daremos una introducción sobre el significado de
monitoreo, explicando dónde radica su importancia, qué información podría ser
útil a los distintos perfiles de usuarios y cómo diseñar un sistema de
monitoreo considerando esa información.

En el \autoref{cap2} explicaremos el caso de estudio al que aplicaremos nuestra
implementación. La metodología de trabajo del laboratorio, las tecnologías que
forman parte de la infraestructura de la aplicación, el estado actual de la
herramienta de monitoreo y nuestros objetivos a la hora de implementar la
solución.

En el \autoref{cap3} explicaremos cómo filtrar y almacenar datos obtenidos en
tiempo real con el objetivo de poder utilizarlos más adelante. Abordaremos la
recolección de datos de red, de recursos de los \gls{term:host}, de los
procesos y de las aplicaciones, entre otros. Además, describiremos las
distintas tecnologías y herramientas usadas.

En el \autoref{cap4} explicaremos cómo implementar la obtención de datos de los
\eng{logs} de las aplicaciones y de los servicios. Describiremos los problemas
comunes en la recolección de datos de \eng{logs} y las herramientas utilizadas
para resolverlos.

En el \autoref{cap5} explicaremos para qué sirve la visualización de datos,
cuestiones importantes a tener en cuenta, la ventaja que da la visualización
para el entendimiento de los datos, herramientas utilizadas y los resultados
obtenidos.

En el \autoref{cap6} explicaremos qué son las alertas, las dificultades de
implementar un buen sistema de alertas y las herramientas utilizadas junto con
los resultados obtenidos.

Finalmente daremos a conocer las conclusiones de la experiencia que hemos
obtenido a lo largo del desarrollo del trabajo y describiremos posibles
trabajos futuros que se pueden realizar tomando como base esta investigación.

En el \autoref{anexo:A} desarrollaremos aspectos de algunas herramientas que
utilizan en el \gls{acro:cespi}.

En el \autoref{anexo:B} mencionaremos otras herramientas que si bien no forman
parte de nuestra solución final, han formado parte de nuestra investigación y
creemos que pueden ser útiles para soluciones similares.

En el \autoref{anexo:C} ampliaremos código utilizado en una parte de nuestra
solución.
