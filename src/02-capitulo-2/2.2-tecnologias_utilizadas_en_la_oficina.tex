\subsection{Tecnologías utilizadas en la oficina}
\label{tecnologias_utilizadas}

La oficina del \gls{acro:cespi} utiliza el \eng{framework} \eng{web}
\gls{term:ror}\footnote{Cf.  \url{http://rubyonrails.org/}} como principal herramienta
de desarrollo. Este framework fue elegido por su excelente documentación,
activa comunidad, facilidad para desarrollar e implementado en el lenguaje
Ruby.

Ruby, como se menciona en su sitio oficial, es un lenguaje creado con el
objetivo de ser “el mejor amigo del desarrollador”. Es orientado a objetos y su
sintaxis es muy amigable en comparación a otros lenguajes de programación. La
expresividad del lenguaje permite escribir código conciso, expresivo y fácil de
mantener.

\gls{term:ror} cuenta con soporte para las distintas bases de datos que se
utilizan en la oficina. Entre ellas se encuentran Mysql[*], MongoDB[*],
Elasticsearch[*] y Redis[*].

La oficina se encuentra en transición a una arquitectura basada en contenedores
de Docker[*].

Docker es un proyecto de código abierto para la automatización del despliegue
de aplicaciones dentro de \glspl{term:contenedor} de software en Linux. Los contenedores
proporcionan una capa adicional de abstracción y automatización de
virtualización a nivel de sistema operativo.

En el marco de la cultura de trabajo \gls{term:devops} del laboratorio se decidió utilizar
Docker para facilitar la construcción de una infraestructura dinámica,
escalable y tolerante a fallos que permitiera el despliegue seguro y automático
de aplicaciones, al mismo tiempo que redujera la brecha entre desarrolladores
de software y profesionales de operaciones en tecnologías de información.

El manejo de múltiples \glspl{term:contenedor} de \gls{term:docker} en producción se convirtió en una
tarea compleja para la oficina, por lo que se decidió utilizar Rancher[*].
Rancher es una plataforma de código abierto que facilita la orquestación,
disponibilidad, configuración y enlace de \glspl{term:contenedor} de Docker. Esta
herramienta permite resolver gran parte de los desafíos críticos presentes al
ejecutar aplicaciones en \glspl{term:contenedor} de Docker.

Rancher provee un conjunto completo de servicios de infraestructura para
contenedores, incluyendo herramientas de red, servicios de almacenamiento de
imágenes de \glspl{term:contenedor} privadas, administración de los hosts que forman parte
de la infraestructura, facilidades para el despliegue, balanceo de carga y
sencillas herramientas de monitoreo.
