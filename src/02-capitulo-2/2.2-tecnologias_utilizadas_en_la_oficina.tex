\subsection{Tecnologías utilizadas en la oficina}
\label{tecnologias_utilizadas}

En sus comienzos, la oficina del \gls{acro:cespi} ha utilizado el
\eng{framework} \eng{web} Symfony 1.4 como herramienta de desarrollo. Hace unos
años comenzó a utilizar \gls{term:ror}\footnote{Cf.
\url{http://rubyonrails.org/}} como principal herramienta de desarrollo. Este
\eng{framework} fue elegido por su excelente documentación, activa comunidad,
facilidad para desarrollar e implementado en el lenguaje \gls{term:ruby}.

\gls{term:ruby}, como se menciona en su sitio oficial, es un lenguaje creado
con el objetivo de ser “el mejor amigo del desarrollador”. Es orientado a
objetos y su sintaxis es muy amigable en comparación a otros lenguajes de
programación. La expresividad del lenguaje permite escribir código conciso,
expresivo y fácil de mantener.

\gls{term:ror} cuenta con soporte para las distintas bases de datos que se
utilizan en la oficina. Entre ellas se encuentran \gls{term:mysql},
\gls{term:mongo}, \gls{term:elasticsearch} y \gls{term:redis}.

Inicialmente tanto la administración de los servidores como la instalación de
las aplicaciones y actualización de las mismas se realizaba en forma manual.
Cuando el número de servidores y aplicaciones comenzó a crecer se analizaron
opciones para automatizar su instalación y actualización.

Entre los años 2011 y 2012 se utilizó \gls{term:capistrano} como herramienta de despliegue
automático.

Luego de in incidente ocurrido en la oficina donde se rompió un componente de
hardware se tuvieron que instalar manualmente todos los servidores con
\gls{term:capistrano}.  Es por este motivo que se evaluaron herramientas de automatización
de la infraestructura y gestión de la misma como código. Es así como hacia el
año 2014 se automatizó el 100\% de la infraestructura y despliegue utilizando
\gls{term:chef}.

Con \gls{term:chef} se definieron 3 ambientes:

\begin{itemize}
  \item \textbf{Testing}: es el ambiente donde se publica el software en fase
    de pruebas para que sea probado por un grupo definido de personas, entre
    las que se incluye el usuario final o representantes del mismo.
  \item \textbf{Pre-producción}: es la instancia previa a producción, y
    consiste en un ambiente replicado e idéntico al productivo. En este entorno
    se verifica el correcto funcionamiento de la aplicación y se realizan los
    ajustes necesarios en caso de no ser ası́, evitando que los problemas se
    descubran en el pasaje a producción.
  \item \textbf{Producción}: es el ambiente que tiene todos los servicios
    productivos. Este ambiente cuenta con polı́ticas estrictas en cuanto al
    acceso y la administración del mismo.
\end{itemize}

El ambiente de \eng{Testing} a diferencia del de Producción y Pre-producción
aún requería el uso de \gls{term:capistrano} por parte de los desarrolladores para
realizar el despliegue de las aplicaciones, lo cual hacía que el ambiente de
\eng{Testing} no sea exactamente igual al del resto.

Con el objetivo de minimizar las diferencias entre los ambientes, la oficina
comenzó la evaluación de la utilización de contenedores \gls{term:docker},
transición que finalizó de forma integra a comienzos del 2017.

\gls{term:docker} es un proyecto de código abierto para la automatización del
despliegue de aplicaciones dentro de \glspl{term:contenedor} de \eng{software}
en \gls{term:linux} y Windows. Los contenedores proporcionan una capa adicional
de abstracción y automatización de virtualización a nivel de sistema operativo.

\gls{term:docker} facilita la construcción de una infraestructura dinámica,
escalable y tolerante a fallos que permite el despliegue seguro y automático
de aplicaciones, al mismo tiempo que reduce la brecha entre desarrolladores de
\eng{software} y profesionales de operaciones en tecnologías de información.

Para lograr que la infraestructura sea realmente escalable y tolerante a fallos
es necesario utilizar un \eng{cluster} de \gls{term:docker}, como por ejemplo:
Docker Swarm, Apache Mesos, Kubernetes y Rancher Cattle.

Se ha decidido emplear \gls{term:rancher} ya que permite crear los distintos
ambientes y definir sobre cada uno de ellos la tecnología de clusterización que
se utilizará.

\gls{term:rancher} es una plataforma de código abierto que facilita la
orquestación, disponibilidad, configuración y enlace de
\glspl{term:contenedor}. Esta herramienta permite resolver gran parte de los
desafíos críticos presentes al ejecutar aplicaciones en \glspl{term:contenedor}
de \gls{term:docker}.

\gls{term:rancher} provee un conjunto completo de servicios para los
contenedores que corren en él, simplificando el \eng{discovery} de servicios,
\eng{scheduling}, \eng{health checking}, despliegue y el escalado de los mismos.
