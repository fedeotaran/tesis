El equipo de trabajo de la oficina desarrolla, mantiene y da soporte a muchas
aplicaciones de diferentes dominios. La mayoría de estas aplicaciones son
implementadas usando los mismos lenguajes, librerías y tecnologías.

Algunas de las aplicaciones del \gls{acro:cespi} incluyen sistemas de manejo de
contenidos, sistemas administrativos de uso interno, un liquidador de sueldo y
otras aplicaciones al servicio de las necesidades de la \gls{acro:unlp}.

El \gls{acro:cespi} mantiene una gran cantidad de aplicaciones, algunas de las
cuales están escritas en el lenguaje de programación \gls{term:php} y otras en
el lenguaje \gls{term:ruby}. El gran número de aplicaciones en producción
complejizaba su mantenimiento, e hizo necesario comenzar a utilizar
herramientas que permitieran el manejo, instalación y aprovisionamiento de los
servidores de forma dinámica, escalable y consistente.

El incentivo del \gls{acro:cespi} por querer mejorar la infraestructura de
monitoreo en sus aplicaciones está estrechamente vinculado con su cultura de
trabajo. En la \autoref{cultura_de_trabajo} se explica en qué consiste esta
forma de trabajo, y de qué forma la aplicación de la cultura \gls{term:devops},
la estrategia \gls{term:lean} y la \gls{term:delivery_continuo} se vinculan con
la necesidad de mejorar su infraestructura de monitoreo de la oficina.

Con el objetivo de contextualizar el trabajo, en la
\autoref{tecnologias_utilizadas} se describen las herramientas y tecnologías
usadas en el \gls{acro:cespi}, se explican las características de la
infraestructura donde se despliegan las aplicaciones y los motivos de porqué
está implementada de esa manera. Además, se muestran algunas de las
particularidades que han sido tenidas en cuenta al diseñar la solución de
monitoreo.

Finalmente en la \autoref{objetivos_de_la_implementacion} se hace mención
las diferentes técnicas de monitoreo que se utilizan actualmente sobre las
aplicaciones y se explica cuales son los objetivos buscados con la
implementación de la arquitectura de monitoreo propuesta.
