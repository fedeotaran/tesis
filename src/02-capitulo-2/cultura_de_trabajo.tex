\subsection{Cultura de trabajo}
\label{cultura_de_trabajo}

La forma de trabajo de la oficina está en constante evolución. Dedica tiempo y
esfuerzo en mejorar la calidad de las tareas que se realizan y la eficiencia en
la que se construyen los productos de software. Hoy en día esta metodología de
trabajo se encuentra fuertemente inspirada en \gls{term:devops}.

\gls{term:devops} es una cultura de trabajo centrada en la comunicación,
colaboración e integración entre desarrolladores de software y profesionales de
operaciones en tecnologías de información. Su objetivo es ayudar a una
organización a producir productos y servicios de software de forma rápida.

El laboratorio cuenta con un enfoque de trabajo de entregas continuas: produce
en cortos y frecuentes ciclos, asegurando que el software pueda ser desplegado
de forma confiable en cualquier momento. Este enfoque ayuda a reducir costos,
tiempo y riesgos de cambios en las entregas al permitir más actualizaciones
incrementales en aplicaciones en producción.

El equipo de trabajo adquiere cada día más prácticas que tengan un enfoque
\gls{term:lean}. \gls{term:lean} es una manera de abordar el lanzamiento de
negocios y productos basada en el aprendizaje validado, experimentación e
iteración en los lanzamientos del producto para acortar los ciclos de
desarrollo, medir el progreso y ganar valiosa retroalimentación de parte de los
clientes.

Uno de los horizontes en cuanto a desarrollo a los que quiere llegar la oficina
es que las aplicaciones recorran el circuito \eng{lean startup} conocido como
“crear, medir y aprender”. Es un proceso iterativo que consiste en transformar
ideas en productos, medir el \eng{feedback} generado por los clientes a partir
del uso de la aplicación y aprender, a partir del análisis de la información
recolectada, para volver a repetir el ciclo.

Esta serie de ciclos comienza con la implementación de un producto mínimo
viable. Esto es, la versión de un nuevo producto que permite al equipo recoger
con el mínimo esfuerzo la máxima cantidad de conocimiento validado por los
consumidores de dicho producto.

La implementación de esta forma de trabajo permite al laboratorio comenzar un
proyecto con un gasto relativamente pequeño y ayudar a los programadores a
iniciar el proceso de aprendizaje sobre la aplicación de la forma más rápida
posible\cite[p~.2]{lean:the_lean_startup}.

Esta forma de trabajo permite que todos los involucrados en el desarrollo del
producto conozcan su ciclo de vida completo. Además incentiva la buena
documentación de la infraestructura y la disponibilidad de métricas para todos.

La filosofía de trabajo ha cumplido un papel fundamental en la promoción del
interés de los programadores y líderes de proyecto de la oficina en llevar a
cabo un monitoreo más adecuado de sus productos y servicios.

Esto es en parte así debido a que la medición constante de los datos de las
aplicaciones y los sistemas en general, demuestra ser invaluable a la hora de
tomar decisiones para contribuir a la mejora continua de dichas aplicaciones y
sistemas. La mejora contínua es uno de los pilares de la metodología de trabajo
de la oficina, y su relación con el monitoreo es una de las razones principales
por las que decidimos comenzar este trabajo.

