\subsection{Objetivo de nuestra implementación}
\label{objetivos_de_nuestra_implementacion}

En este contexto es importante construir una infraestructura de monitoreo que
permita obtener valiosa información sobre las reglas de negocio de las
aplicaciones que se desarrollan en la oficina, de forma que permita al equipo
de trabajo contar con datos duros para tomar decisiones de negocio más
confiables.

Además, los desarrolladores del \gls{acro:cespi} creen que sería útil contar
con una herramienta de recolección y análisis de información sobre el
rendimiento de las aplicaciones. Esta herramienta podría ayudar a encontrar
puntos débiles y cuellos de botella en las aplicaciones.

A las personas encargadas de mantener la infraestructura de la oficina les
interesa tener un seguimiento de datos de uso del sistema operativo y las
computadoras, como el uso de \gls{acro:cpu}, la cantidad de memoria utilizada y
disponible, número de tareas en espera, uso de la memoria \gls{term:swap},
cantidad de procesos, espacio utilizado en disco.

Anteriormente los desarrolladores no contaban con una implementación formal de
monitoreo, sino que recurrían a diversas técnicas para llevar a cabo el
seguimiento de las aplicaciones, detectar errores y prevenir su aparición.

Una de las técnicas utilizadas por el grupo de desarrollo era la búsqueda y
lectura de registros de \eng{logs} de forma manual. El programador se conectaba
a través de \gls{term:ssh} al servidor de la aplicación y visualizaba los
\eng{logs} directamente mediante la consola.

Si el desarrollador requería conocer información en tiempo real de los recursos
del servidor donde estaban corriendo las aplicaciones, éste debía conectarse al
servidor y ver la información utilizando el comando \gls{term:htop}.

Para obtener una devolución del funcionamiento de las consultas a bases de
datos se utilizaba una herramienta que permitía configurar el envío de reportes
a través de correo electrónico usando la funcionalidad
\texttt{slow\_query\_log} provista por \gls{term:mysql}. Esta funcionalidad
guarda información de consultas que toman mucho tiempo en llevarse a cabo. Este
tiempo es por defecto de 10 segundos, pero puede ser definido por un usuario.

Para ser notificados de errores en producción de las aplicaciones
\gls{term:ror}, los programadores hacían uso de la \gls{term:gema}
\eng{Exception Notifier}\footnote{Cf.
\url{https://github.com/smartinez87/exception_notification}}. Esta librería
permite configurar en el código de la aplicación, una dirección de correo para
recibir las notificaciones de errores en la aplicación, y se acciona cuando la
aplicación devuelve a un usuario un código \gls{acro:http} de error.

La notificación cuenta con valiosa información del contexto de la aplicación.
Algunos ejemplos son el tipo de solicitud, los datos de la sesión del usuario,
los parámetros de la petición y también el \gls{term:stack_trace}, que contiene
información de las llamadas a los métodos que dieron origen al error.

Para realizar recolección, análisis de datos y generación de alertas, en la
oficina se utiliza las herramientas de monitoreo \gls{term:graphite} y
\gls{term:nagios}.

\gls{term:graphite} es una herramienta libre y de código abierto que sirve para
monitorear y graficar el rendimiento de sistemas de computadoras.
\gls{term:graphite} recolecta, almacena y muestra datos de tipo series de
tiempo en tiempo real.

\gls{term:nagios} es una aplicación libre y de código abierto que monitoreo
sistemas, redes e infraestructura. \gls{term:nagios} ofrece monitoreo y
servicios de alertas para servidores, \eng{switches}, aplicaciones y servicios.

Una desventaja del uso de estos mecanismos es que generan una notificación cada
vez que ocurre un error. Por lo que si un error ocurre cientos de veces en un
día, el usuario recibiría cientos de mensajes idénticos, cuando con avisar una
vez del error seguramente era suficiente.

Partiendo del conocimiento que se obtuvo con la investigación, puede ver la
importancia de mejorar la forma en que se implementaba el monitoreo en el
\gls{acro:cespi}.

La oficina podría beneficiarse con una infraestructura de monitoreo que brinde
información útil a los usuarios, sea fácil de utilizar, pueda adaptarse a una
arquitectura de \glspl{term:contenedor} y sea dinámica y escalable.

El primer paso para implementar la nueva infraestructura de monitoreo será
recolectar métricas en tiempo real que provengan del uso de las aplicaciones y
los \glspl{term:contenedor}, y obtener de ellos valiosa información que nos
permita prevenir errores y tomar mejores decisiones de diseño.

Luego hacer uso de la información que brindan los \eng{logs} de forma más
intensiva. Unificar el formato de los \eng{logs} y centralizar los mismo que
provengan de diversas fuentes en un único lugar.

La infraestructura de aplicaciones y servicios que se quiere monitorear es
dinámica, escalable, construida de forma automatizada y creada a partir de
código reutilizable. Lograr una forma de monitoreo que pueda adaptarse a las
características de esta infraestructura es lo que se busca, por lo que habrá
que investigar cómo utilizar \glspl{term:contenedor} de \gls{term:docker} para
implementar algunos componentes del sistema de monitoreo.

Con el objetivo de impulsar un sistema que brinde información útil a los
usuarios, se va a aplicar algunas de las técnicas de visualización efectiva
mencionadas en el capítulo anterior (\autoref{visualizacion-efectiva}) durante
la creación de tableros con gráficos.

Finalmente, mostrar la propuesta de una implementación que de una solución de
alertas de mayor utilidad que evite, en la medida de lo posible, falsos
positivos y negativos y brinde notificaciones útiles a los usuarios indicando
concretamente el problema y/o ayudándolos a prevenirlos.

En los próximos capítulos se expone la forma en que desarrolló una
infraestructura de monitoreo que intente cumplir con las características
mencionadas.
