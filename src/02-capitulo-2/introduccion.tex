El equipo de trabajo de la oficina desarrolla, mantiene y da soporte a muchas
aplicaciones de diferentes dominios. La mayoría de estas aplicaciones son
implementadas usando los mismos lenguajes, librerías y tecnologías.

Algunas de las aplicaciones del laboratorio incluyen liquidadores de sueldos,
sistemas de gestión de contenidos, sistemas de uso interno del área
administrativa de la facultad y otras aplicaciones al servicio de las
necesidades de la \gls{acro:unlp}.

El laboratorio mantiene una gran cantidad de aplicaciones, algunas de las
cuales están escritas en el lenguaje de programación \gls{term:php} y otras en
el lenguaje \gls{term:ruby}. El gran número de aplicaciones en producción
complejizaba su mantenimiento, e hizo necesario comenzar a utilizar
herramientas que permitieran el manejo, instalación y aprovisionamiento de los
servidores de forma dinámica, escalable y consistente.

El incentivo del laboratorio por querer mejorar la infraestructura de monitoreo
en sus aplicaciones está estrechamente vinculado con su cultura de trabajo. En
la sección 2.1 explicaremos en qué consiste esta forma de trabajo, y de qué
forma la aplicación de la cultura \gls{term:devops}, la estrategia
\gls{term:lean} y la \gls{term:delivery_continuo} se vinculan con la necesidad
de mejorar su infraestructura de monitoreo de la oficina.

Con el objetivo de contextualizar el trabajo, en la sección 2.2 describiremos
las herramientas y tecnologías usadas en el laboratorio, se explicarán las
características de la infraestructura donde se despliegan las aplicaciones y
los motivos de porqué está implementada de esa manera. Además, vamos a exponer
algunas de las peculiaridades que han sido tenidas en cuenta al diseñar nuestra
solución de monitoreo.

Finalmente en la sección 2.3 describiremos la forma en que está implementado
actualmente el monitoreo de la aplicación y explicaremos cuales son los
objetivos que buscamos con la implementación de nuestra arquitectura de
monitoreo.
