En este capítulo explicaremos cómo hemos logrado recolectar, almacenar y
consultar información generada en tiempo real de varias fuentes. Ejemplos de
esta información son el uso del disco rígido, los procesadores y la memoria.

En la \autoref{almacenamiento} repasaremos las características que tienen los
datos generados en tiempo real, y describiremos brevemente algunas herramientas
que permiten el almacenamiento y consulta de estos datos de forma eficiente. En
particular explicaremos qué es \gls{term:influx} y cómo configurarlo.

En la \autoref{aplicaciones} demostraremos cómo obtener información valiosa de
las aplicaciones \gls{term:ror} a partir de la instrumentación y daremos una
introducción a la librería \texttt{influxdb-rails}, que nos permitirá enviar
datos tomados de las aplicaciones a nuestra instancia de \gls{term:influx}.

En la \autoref{contenedores} mostraremos herramientas que nos permitan obtener
información del funcionamiento en tiempo real de contenedores de
\gls{term:docker}. En particular desarrollaremos cómo configurar la herramienta
\gls{term:cadvisor} para enviar información de los contenedores a
\gls{term:influx}.
