En este capítulo se explicará cómo se ha logrado recolectar, almacenar y
consultar información generada en tiempo real de varias fuentes. Ejemplos de
esta información son el uso del disco rígido, los procesadores y la memoria.

En la \autoref{almacenamiento} se repasarán las características que tienen los
datos generados en tiempo real, y se describirán brevemente algunas herramientas
que permiten el almacenamiento y consulta de estos datos de forma eficiente. En
particular se explicará qué es \gls{term:influx} y cómo usarlo.

En la \autoref{aplicaciones} se demostrará cómo obtener información valiosa de
las aplicaciones \gls{term:ror} a partir de la instrumentación y se dará una
introducción a la librería \texttt{influxdb-rails}, que permitirá enviar datos
tomados de las aplicaciones a la instancia de \gls{term:influx}.

En la \autoref{contenedores} se mostrarán herramientas que nos permitan obtener
información del funcionamiento en tiempo real de \glspl{term:contenedor}
\gls{term:docker}. En particular se desarrollará cómo configurar la herramienta
\gls{term:cadvisor} para enviar información de los \glspl{term:contenedor} a
\gls{term:influx}.
