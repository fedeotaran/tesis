En este capítulo explicaremos cómo hemos logrado recolectar, almacenar y
consultar información generada en tiempo real de varias fuentes. Ejemplos de
esta información son el uso del disco rígido, los procesadores y la memoria.

En la sección 3.1 repasaremos las características que tienen los datos generados
en tiempo real, y describiremos brevemente algunas herramientas que permiten el
almacenamiento y consulta de estos datos de forma eficiente. En particular
explicaremos qué es InfluxDB y cómo configurarlo.

En la sección 3.2 demostraremos cómo obtener información valiosa de las
aplicaciones \gls{term:ror} a partir de la instrumentación y daremos una
introducción a la librería influxdb-rails, que nos permitirá enviar datos
tomados de las aplicaciones a nuestra instancia de InfluxDB.

En la sección 3.3 mostraremos herramientas que nos permitan obtener información
del funcionamiento en tiempo real de contenedores de \gls{term:docker}. En
particular desarrollaremos cómo configurar la herramienta cAdvisor para enviar
información de los contenedores a InfluxDB.

En la sección 3.4 mostraremos cómo se pueden utilizar las herramientas que hemos
configurado a lo largo del capítulo para almacenar y consultar datos a InfluxDB
a través de su cliente web.
