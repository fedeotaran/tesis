\subsection{Recolección de datos de aplicaciones}
\label{aplicaciones}

En  esta sección se explicará cómo enviar datos de las aplicaciones \eng{web} a
\gls{term:influx}. Para enviar datos se ha usado la \gls{term:gema}
\texttt{influxdb-rails}. Esta \gls{term:gema} proporciona métodos para vincular
\gls{term:ror} con \gls{term:influx}.

Para que la aplicación \gls{term:ror} pueda hacer uso de estos métodos, es
necesario agregar la siguiente línea al archivo \texttt{Gemfile}:

\begin{lstlisting}

gem "influxdb-rails"

\end{lstlisting}

Una vez que se tiene instalada la \gls{term:gema}, es posible configurar el
adaptador \texttt{InfluxDB::Rails} en un archivo de configuración de la
aplicación. La configuración por defecto debería verse de la siguiente manera:

\begin{lstlisting}
InfluxDB::Rails.configure do |config|
  config.influxdb_database = "rails"
  config.influxdb_username = "root"
  config.influxdb_password = "root"
  config.influxdb_hosts    = ["influx"]
  config.influxdb_port     = 8086
end
\end{lstlisting}

El código \gls{term:ruby} anterior configura los datos necesarios para que la
aplicación pueda comunicarse con la base de datos. Estos datos son el nombre de
la base de datos, el nombre de usuario, la contraseña del usuario, el nombre del
\gls{term:host} y el número de puerto por el cual comunicarse con la
\gls{term:api} de \gls{term:influx}.

Sin necesidad de configurar nada más, la \gls{term:gema} brinda automáticamente
reportes de los tiempos de ejecución de cada solicitud \eng{web} para los
controladores, las vistas y las bases de datos. Pero además, posibilita la
realización de llamadas al cliente \gls{term:influx} directamente, escribiendo
datos arbitrarios de la siguiente manera:

\begin{lstlisting}
InfluxDB::Rails.client.write_point "events",
  tags:   { url: "/ejemplo", user_id: 1 },
  values: { value: 0 }
\end{lstlisting}

A continuación se mostrará un ejemplo concreto de cómo se puede utilizar esta
herramienta para generar métricas útiles para el equipo de desarrollo y para
los dueños de un sistema:

La \gls{acro:unlp} brinda un servicio para que los alumnos obtengan su libreta
sanitaria que permite registrar las consultas médicas que realicen y solicitar
turnos para ir a los consultorios médicos. Este sistema consiste en una
aplicación desarrollada en el \eng{framework} \gls{term:ror}.

A la gerencia le parece importante mantener un seguimiento de los inicios de
sesión de usuarios en la aplicación. Es posible utilizar la librería
\texttt{influxdb-rails} para cumplir este objetivo, escribiendo el siguiente
código \gls{term:ruby} para que sea ejecutado cada vez que un usuario inicia
sesión de forma exitosa:

\begin{lstlisting}
InfluxDB::Rails.client.write_point(
  "logins",
  tags:   {
    ip: request.remote_ip,
    user_id: current_user.id,
    academic_unit: current_user.academic_unit
  },
  values: { value: request_time }
)
\end{lstlisting}

Las expresiones \lstinline{request.remote_ip} y \lstinline{current_user.id}
contienen respectivamente la dirección \gls{acro:ip} y el identificador único
del usuario que se encuentra iniciando sesión en alguno de los sistemas de la
\gls{acro:unlp}. La dependencia a la cual pertenece el usuario es obtenida a
través de \lstinline{current_user.academic_unit}. El tiempo quedemoró la
petición \eng{web} en resolverse se puede encontrar en la variable
\lstinline{request_time}.

A partir de esta información que se está enviando a \gls{term:influx}, es
posible obtener varias métricas que pueden ser útiles para los desarrolladores,
profesionales de \gls{acro:it} y otros. Algunos ejemplos de estas métricas son:

\begin{itemize}
  \item tiempo de respuesta promedio para la operación de inicio de sesión
  \item promedio de accesos por cada dependencia
  \item accesos de cada dependencia agrupados por franja horaria
\end{itemize}

Este tipo de información puede ayudar a guiar decisiones dentro de las
distintas áreas del \gls{acro:cespi}.

\gls{term:ror} incluye una \gls{term:api} de instrumentación
\footnote{Guía oficial de instrumentación:
\url{http://guides.rubyonrails.org/active_support_instrumentation.html}}
que sirve para medir eventos. Con esta \gls{term:api}, los desarrolladores
pueden elegir ser notificados cuando ciertas acciones ocurren dentro de las
aplicaciones.

La \gls{term:api} de instrumentación provee enganches, o \eng{hooks} a los
cuales los desarrolladores pueden suscribirse y así ser notificados en caso de
que determinados eventos se produzcan. De esta forma, se podrán tomar acciones
a partir de la ocurrencia de estos eventos. Además, la \gls{term:api} provee
herramientas para que un programador pueda crear sus propios \eng{hooks} a los
cuales subscribirse.

Por ejemplo, existe un \eng{hook} provisto por la librería de \gls{term:ror},
\eng{Active Record}\footnote{Guía oficial de Active Record:
\url{http://guides.rubyonrails.org/active_record_basics.html}}, que es
llamado cada vez que se lleva a cabo una consulta \gls{acro:sql} en una base de datos. Es
posible subscribirse a este \eng{hook}, y utilizarlo para tener un seguimiento
del número de consultas realizadas.

Podemos aprovechar las facilidades que provee la \gls{term:api} de
instrumentación de \gls{term:ror} para enviar a \gls{term:influx} información
sobre las consultas que se hacen a la base de datos, agregando las siguientes
líneas a un archivo de configuración:

\begin{lstlisting}
ActiveSupport::Notifications.subscribe("sql.active_record") do |name, start, finish, id, payload|
  InfluxDB::Rails.client.write_point name,
    value: (finish-start),
    start: start,
    finish: finish,
    name: name
end
\end{lstlisting}

La \gls{term:gema} \texttt{influxdb-rails} viene configurada por defecto para
apoyarse en el \eng{hook} \lstinline{process_action.action_controller} y
enviar a \gls{term:influx} información sobre los tiempos de respuesta de cada
petición \eng{web}, distinguiendo el tiempo que demora en el \eng{renderizado}
de las vistas, el procesamiento de los controladores y las consultas a las bases
de datos.

Para ejecutar la aplicación \gls{term:ror} usando \glspl{term:contenedor} de
\gls{term:docker}, se agregará el archivo \gls{term:dockerfile} en el directorio
raíz de la aplicación:

\begin{lstlisting}
FROM ruby:2.3.1

RUN apt-get update -qq && apt-get install -y build-essential libpq-dev nodejs sqlite libsqlite3-dev

RUN mkdir /app

WORKDIR /app

COPY Gemfile Gemfile.lock ./
RUN gem install bundler && bundle install

COPY . ./

EXPOSE 3000

# Configure an entry point, so we don't need to specify
# "bundle exec" for each of our commands.
ENTRYPOINT ["bundle", "exec"]

CMD ["rails", "server", "-b", "0.0.0.0"]
\end{lstlisting}

La línea \lstinline{FROM ruby:2.3.1} declara que el \gls{term:contenedor} se
basa en la imagen de \gls{term:docker} de nombre \lstinline{ruby:2.3.1}. La
siguiente lìnea se ocupa de instalar las librerías necesarias para usar la
aplicación \gls{term:ror}.

Luego se crea la carpeta de la aplicación, se copian los archivos
\lstinline{Gemfile} y \lstinline{Gemfile.lock}, donde se describen las
dependencias de la aplicación, y se instalan las \glspl{term:gema} de esos
archivos con el comando \lstinline{bundle install}.

Finalmente, se copia el contenido de la aplicación al \gls{term:contenedor}
\gls{term:docker}, se expone el puerto 3000, y se corre la aplicación con el
comando \lstinline{rails server -b 0.0.0.0}. En la \autoref{anexo:docker}
se explica un poco mas de esta herarmienta.

Se mostrará a continuación cómo se puede hacer para correr la aplicación al
mismo tiempo que se ejecutan sus servicios. Para esta tarea se utilizará
Compose de \gls{term:docker}. Compose permite definir una
configuración de los servicios de una aplicación en un archivo, e iniciar todos
estos servicios con un sólo comando.

Se ha escrito una explicación un poco más detallada de Compose en el
\autoref{anexo_compose}.

El siguiente archivo \gls{term:docker} Compose es un ejemplo de cómo configurar
la aplicación \gls{term:ror} para trabajar con \gls{term:influx}.

\begin{lstlisting}
version: "2"
services:
  app:
    build: app
    command: rails server -p 3000 -b '0.0.0.0'
    volumes:
      - ./app/:/app
    ports:
      - "3000:3000"
    networks:
      - metricnet

  influxsrv:
    image: influxdb:1.0.0-rc1
    ports:
      - "8083:8083"
      - "8086:8086"
    expose:
      - "8090"
      - "8099"
    networks:
      - metricnet

networks:
  metricnet:
    driver: bridge
\end{lstlisting}

En el archivo se definen dos servicios. La aplicación \gls{term:ror} es el
servicio de nombre \lstinline{app}, y la base de datos \gls{term:influx} es el
servicio de nombre \lstinline{influx}. Ambos utilizan la red \lstinline{lognet}
para comunicarse entre sí.

\lstinline{app} usa el \gls{term:dockerfile} que se describió anteriormente para
inicializar el servicio de la aplicación, y define los puertos externo e interno
\lstinline{3000:3000}.

\lstinline{influx} utiliza la imagen \lstinline{influxdb:1.0.0-rc1} y configura
algunas variables de ambiente, como lo son el nombre del usuario administrador,
la contraseña del usuario y el nombre de la base de datos a crear por defecto.
Finalmente define los puertos por donde estará la \gls{term:api} y el cliente
\eng{web}.
