\subsection{Recolección de datos de contenedores Docker}
\label{contenedores}

Como hemos mencionado anteriormente, el laboratorio usa \glspl{term:contenedor}
de \gls{term:docker} en su infraestructura. Mientras estos
\glspl{term:contenedor} están en ejecución, es posible tomar información de su
funcionamiento en tiempo real.

Para realizar la recolección de datos de los \glspl{term:contenedor}, donde se
mantienen en ejecución las aplicaciones, hemos investigado las herramientas
\gls{term:cadvisor} y Telegraf.

\gls{term:cadvisor} es un demonio que recolecta, agrega,
procesa y exporta información de \glspl{term:contenedor} en ejecución.
Específicamente, para cada \gls{term:contenedor} guarda parámetros de
aislamiento de recursos, uso de recursos históricos, histogramas de uso
completo de recursos históricos y estadísticas de red. Estos datos se exportan
para la máquina host donde está corriendo y para cada uno de los
\glspl{term:contenedor}.

\gls{term:cadvisor} facilita a los usuarios el entendimiento del uso de
recursos y las características de rendimiento de los \glspl{term:contenedor} en
ejecución.  Tiene soporte nativo para \glspl{term:contenedor} \gls{term:docker}
y puede utilizarse con casi cualquier otro tipo de \gls{term:contenedor}.
\footnote{\url{https://github.com/google/cadvisor}}

Telegraf es un agente escrito en el lenguaje de programación \gls{term:go} para
recolectar, procesar, agregar y escribir métricas. Fue diseñado con el objetivo
de requerir poca memoria, y de proveer un sistema de \eng{plugins} de forma que
los desarrolladores de la comunidad pudieran fácilmente agregar soporte para
recolección de métricas de diferentes servicios y \glspl{term:api}.

Telegraf utiliza cuatro conceptos distintos de \eng{plugins}:

\begin{itemize}

  \item \textbf{Input Plugins:}
  Recolectan métricas del sistema, servicios o \glspl{term:api} de terceros.

  \item \textbf{Processor Plugins:}
  Transforman, decoran y filtran métricas.

  \item \textbf{Aggregator Plugins:}
  Crean métricas de agregación, como promedio, mínimo o máximo.

  \item \textbf{Output Plugins:}
  Escriben métricas a varios destinos.

\end{itemize}

Finalmente hemos elegido la herramienta \gls{term:cadvisor} para nuestra
implementación por sobre Telegraf porque la primera tenía soporte nativo para
\glspl{term:contenedor} de \gls{term:docker}, lo cual consideramos una ventaja.

La recomendación oficial para utilizar \gls{term:cadvisor} es ejecutarlo desde
un \gls{term:contenedor} de \gls{term:docker} a partir de la imágen oficial
\footnote{\url{https://hub.docker.com/r/google/cadvisor/}}. Con el siguiente
comando se puede iniciar la ejecución de la imagen mencionada para monitorizar
los recursos de una máquina y todos los \glspl{term:contenedor} corriendo en la
misma:

\begin{lstlisting}
  sudo docker run \
    --volume=/:/rootfs:ro \
    --volume=/var/run:/var/run:rw \
    --volume=/sys:/sys:ro \
    --volume=/var/lib/docker/:/var/lib/docker:ro \
    --publish=8080:8080 \
    --detach=true \
    --name=cadvisor \
    google/cadvisor:latest
\end{lstlisting}


Además \gls{term:cadvisor} permite exportar estadísticas a distintos tipos de
herramientas de almacenamiento por medio de \eng{plugins}. Las herramientas que
tienen integración con \gls{term:cadvisor} son:

\begin{itemize}
  \item BigQuery
  \item \gls{term:elasticsearch}
  \item \gls{term:influx}
  \item Kafka
  \item Prometheus
  \item Redis
  \item StatsD
\end{itemize}

Además de las herramientas permite la salida de información a la
\gls{acro:stdout}.

Utilizaremos un plugin para permitir a \gls{term:cadvisor} enviar la
información recolectada a \gls{term:influx}. Para configurar la comunicación
con la instancia de \gls{term:influx} que se desea utilizar, se dispone de las
siguiente opciones
\footnote{\url{https://github.com/google/cadvisor/blob/master/docs/storage/influxdb.md}}:

\begin{lstlisting}
  # Elección de manejador de almacenamiento
  -storage_driver=influxdb

  # IP y puerto de la base de datos. Por defecto es 'localhost:8086'
  -storage_driver_host=ip:port

  # Nombre de la base de datos. Por defecto es 'cadvisor'
  -storage_driver_db

  # Usuario de la base de datos. Por defecto es 'root'
  -storage_driver_user

  # Contraseña de la base de datos. Por defecto es 'root'
  -storage_driver_password

  # Indicador de conexión segura con la base de datos. Falso por defecto.
  -storage_driver_secure
\end{lstlisting}

Para configurar el servicio de \gls{term:cadvisor} agregamos las siguientes
líneas al archivo \lstinline{docker-compose.yml}:

\begin{lstlisting}
  cadvisor:
    image: google/cadvisor:v0.24.0
    command: -storage_driver=influxdb \
             -storage_driver_db=cadvisor \
             -storage_driver_host=influxsrv:8086
    ports:
      - "9090:8080"
    volumes:
      - /:/rootfs:ro
      - /var/run:/var/run:rw
      - /sys:/sys:ro
      - /var/lib/docker/:/var/lib/docker:ro
    networks:
      - lognet
\end{lstlisting}

Con estas configuraciones logramos recolectar los datos de los
\glspl{term:contenedor} \gls{term:docker} y configurar \gls{term:cadvisor} para
enviarlos a nuestra base de datos de series de tiempo.
