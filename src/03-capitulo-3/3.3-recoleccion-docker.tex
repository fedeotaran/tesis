\subsection{Recolección de datos de contenedores Docker}
\label{contenedores}

Como hemos mencionado anteriormente, el laboratorio usa contenedores de Docker
en su infraestructura. Mientras estos contenedores están en ejecución, es
posible tomar información de su funcionamiento en tiempo real.

Para realizar la recolección de datos de los contenedores, donde se mantienen
en ejecución las aplicaciones, hemos investigado las herramientas cAdvisor y
Telegraf. 

cAdvisor (Container Advisor) es un demonio que recolecta, agrega, procesa y
exporta información de contenedores en ejecución. Específicamente, para cada
contenedor guarda parámetros de aislamiento de recursos, uso de recursos
históricos, histogramas de uso completo de recursos históricos y estadísticas
de red. Estos datos se exportan para la máquina host donde está corriendo y
para cada uno de los contenedores.

cAdvisor facilita a los usuarios el entendimiento del uso de recursos y las
características de rendimiento de los contenedores en ejecución. Tiene soporte
nativo para contenedores Docker y puede utilizarse con casi cualquier otro tipo
de contenedor. \footnote{\url{https://github.com/google/cadvisor}}

Telegraf es un agente escrito en el lenguaje de programación Go para
recolectar, procesar, agregar y escribir métricas. Fue diseñado con el objetivo
de requerir poca memoria, y de proveer un sistema de plugins de forma que los
desarrolladores de la comunidad pudieran fácilmente agregar soporte para
recolección de métricas de diferentes servicios y APIs.

Telegraf utiliza cuatro conceptos distintos de plugins:

\begin{itemize}

  \item \textbf{Input Plugins:}
  Recolectan métricas del sistema, servicios o APIs de terceros.

  \item \textbf{Processor Plugins:}
  Transforman, decoran y filtran métricas.

  \item \textbf{Aggregator Plugins:}
  Crean métricas de agregación, como promedio, mínimo o máximo.

  \item \textbf{Output Plugins:}
  Escriben métricas a varios destinos.

\end{itemize}

Finalmente hemos elegido la herramienta cAdvisor para nuestra implementación por sobre Telegraf porque la primera tenía soporte nativo para contenedores de Docker, lo cual consideramos una ventaja.

La recomendación oficial para utilizar cAdvisor es ejecutarlo desde un contenedor de Docker a partir de la imágen oficial \footnote{\url{https://hub.docker.com/r/google/cadvisor/}}. Con el siguiente comando se puede iniciar la ejecución de la imagen mencionada para monitorizar los recursos de una máquina y todos los contenedores corriendo en la misma:

\begin{lstlisting}
sudo docker run \
  --volume=/:/rootfs:ro \
  --volume=/var/run:/var/run:rw \
  --volume=/sys:/sys:ro \
  --volume=/var/lib/docker/:/var/lib/docker:ro \
  --publish=8080:8080 \
  --detach=true \
  --name=cadvisor \
  google/cadvisor:latest
\end{lstlisting}


Además cAdvisor permite exportar estadísticas a distintos tipos de herramientas de almacenamiento por medio de plugins. Las herramientas que tienen integración con cAdvisor son:

\begin{itemize}
  \item BigQuery
  \item ElasticSearch
  \item InfluxDB
  \item Kafka
  \item Prometheus
  \item Redis
  \item StatsD
\end{itemize}

Además de las herramientas permite la salida de información a la salida estandar (STDOUT).

Utilizaremos un plugin para permitir a cAdvisor enviar la información recolectada a InfluxDB. Para configurar la comunicación con la instancia de InfluxDB que se desea utilizar, se dispone de las siguiente opciones: \footnote{\url{https://github.com/google/cadvisor/blob/master/docs/storage/influxdb.md}}:

\begin{lstlisting}
# Elección de manejador de almacenamiento
 -storage_driver=influxdb

# IP y puerto de la base de datos. Por defecto es 'localhost:8086'
 -storage_driver_host=ip:port

# Nombre de la base de datos. Por defecto es 'cadvisor'
 -storage_driver_db

# Usuario de la base de datos. Por defecto es 'root'
 -storage_driver_user

# Contraseña de la base de datos. Por defecto es 'root'
 -storage_driver_password

# Indicador de conexión segura con la base de datos. Falso por defecto.
 -storage_driver_secure
\end{lstlisting}

Para configurar el servicio de cAdvisor agregamos las siguientes líneas al archivo \lstinline{docker-compose.yml}:

\begin{lstlisting}
  cadvisor:
    image: google/cadvisor:v0.24.0
    command: -storage_driver=influxdb \
             -storage_driver_db=cadvisor \
             -storage_driver_host=influxsrv:8086
    ports:
      - "9090:8080"
    volumes:
      - /:/rootfs:ro
      - /var/run:/var/run:rw
      - /sys:/sys:ro
      - /var/lib/docker/:/var/lib/docker:ro
    networks:
      - lognet
\end{lstlisting}

Con estas configuraciones logramos recolectar los datos de los contenedores Docker y configurar cAdvisor para enviarlos a nuestra base de datos de series de tiempo.
