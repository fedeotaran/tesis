\subsection{Almacenamiento}
\label{almacenamiento}

En la sección 1.2 hemos descrito las particularidades que tienen los datos de
series de tiempo. Una de estas particularidades es que todas las series de
tiempo comparten entre sí una dimensión: la dimensión de tiempo, y por eso son
útiles para correlacionar datos de varias fuentes.

Las bases de datos relacionales tradicionales pueden no ser prácticas para
manejar almacenar estos tipos de datos. Una base de datos de series de tiempo es
un sistema de software que está optimizado para manejarlos de forma correcta,
confiable y eficiente.

Hemos analizado las herramientas InfluxDB y Graphite para almacenar datos de
este tipo. 

InfluxDB es una base de datos de código abierto implementada en el lenguaje Go
con el propósito de manejar datos de series de tiempo con requerimientos de gran
disponibilidad y alto rendimiento. InfluxDB no tiene dependencias externas.

InfluxDB viene con una API HTTP integrada, lo que permite que no sea necesario
escribir ningún código del lado de servidor para comenzar a trabajar.
\footnote{\url{https://github.com/influxdata/influxdb}}

Es posible etiquetar los datos, lo que permite una consulta muy flexible. Esta
consulta se escribe en un lenguaje similar a SQL, llamado InfluxQL. \footnote{
  \url{https://docs.influxdata.com/influxdb/v0.9/query_language/query_syntax/}
}

InfluxDB permite contestar consultas en tiempo real, lo que significa que cada
valor se indexa a medida que llega y está disponible casi inmediatamente para su
uso.

Graphite es una herramienta de monitoreo capaz de ser ejecutada en una amplia
gama de computadoras, y también en la nube. Graphite es usado para tener un
seguimiento del rendimiento de sitios web, aplicaciones, servicios de negocio y
servidores en una red.

Con Graphite es relativamente sencillo almacenar, recuperar, compartir y
visualizar datos de tipo series de tiempo.

Graphite consiste en tres componentes de software:

\begin{itemize}

  \item \textbf{carbon:}
  Un servicio de alto rendimiento que recepta datos de tipo series de tiempo

  \item \textbf{whisper:}
  Una base de datos sencilla para almacenar datos de tipo series de tiempo

  \item \textbf{graphite-web:}
  Una interfaz de usuario y API para visualizar gráficos y tableros.

\end{itemize}

Las métricas son alimentadas a través del servicio Carbon, que envía datos a las
bases de datos Whisper para su almacenamiento a largo plazo. Los usuarios
interactúan con la interfaz de graphite-web o con su API, que a su vez consulta
a Carbon y Whisper por los datos necesarios para construir los gráficos
requeridos.


La plataforma web de Graphite ofrece una variedad de estilos y formatos de
salidas, incluyendo imágenes, archivos separados por comas, XML y JSON.
\footnote{
  \url{http://graphite.readthedocs.io/en/latest/overview.html}
}

OpenTSDB es una base de datos de series de tiempos distribuida y escalable,
escrita sobre HBase. OpenTSDB almacena, indexa y sirve métricas recolectadas de
sistemas de computadoras, como por ejemplo redes, sistemas operativos y
aplicaciones, a gran escala, y hace que estos datos sean accesibles y
graficables de forma fácil.

OpenTSDB permite recolectar métricas de hosts y aplicaciones a una taza de
tiempo alta y es capaz de recolectar miles de métricas de decenas de miles de
hosts y aplicaciones, y almacenar miles de millones de valores estadísticos.

OpenTSDB es software libre y está disponible en ambas licencias LGPLv2.1 + GPLv3
\footnote{Find more about OpenTSDB at \url{http://opentsdb.net}}

Finalmente hemos elegido InfluxDB para nuestra implementación por sobre las
demás herramientas. La decisión no fue fácil, ya que todas las herramientas nos
parecieron muy prometedoras. La razón por la que elegimos InfluxDB es que nos
pareció que esta herramienta tenía mejor soporte para el lenguaje Ruby que las
alternativas. Además, creímos que el hecho de que InfluxDB tenga un lenguaje de
consulta similar a SQL facilitaría el desarrollo de consultas al personal de la
oficina.

InfluxDB es fácil de instalar y administrar.

Puede ser instalado usando uno de sus paquetes pre armados para diferentes
sistemas operativos. \footnote{\url{https://portal.influxdata.com/downloads}}.
Para instalar InfluxDB versión 1.2 en una distribución Linux basada en Debian,
basta con correr los siguientes comandos en una terminal:

\begin{lstlisting}
wget https://dl.influxdata.com/influxdb/releases/influxdb_1.2.0_amd64.deb
sudo dpkg -i influxdb_1.2.0_amd64.deb
\end{lstlisting}

Luego, para ejecutar el programa basta con correr el comando service influxdb
start.

Una vez que InfluxDB está instalado, es posible crear una base de datos,
insertar datos y realizar consultas usando la API de InfluxDB. Incluso se puede
hacer todo esto desde la consola usando el comando curl:

1- Crear una base de datos de nombre monitoreo:

\begin{lstlisting}
curl -XPOST 'http://localhost:8086/query' --data-urlencode "q=CREATE DATABASE monitoreo"
\end{lstlisting}

2- Insertar algunos datos:

\begin{lstlisting}
curl -XPOST 'http://localhost:8086/write?db=monitoreo \
-d 'cpu,host=server01,region=uswest load=42 1434055562000000000'

curl -XPOST 'http://localhost:8086/write?db=monitoreo \
-d 'cpu,host=server02,region=uswest load=78 1434055562000000000'

curl -XPOST 'http://localhost:8086/write?db=monitoreo \
-d 'cpu,host=server03,region=useast load=15.4 1434055562000000000'
\end{lstlisting}


3- Consultar los datos:

\begin{lstlisting}
curl -G http://localhost:8086/query?pretty=true --data-urlencode "db=monitoreo" \
--data-urlencode "q=SELECT * FROM cpu WHERE host='server01' AND time < now() - 1d"

curl -G http://localhost:8086/query?pretty=true --data-urlencode "db=monitoreo" \
--data-urlencode "q=SELECT mean(load) FROM cpu WHERE region='uswest'"
\end{lstlisting}

InfluxDB utiliza varios puertos de red para funcionar. Todos los mapeos de
puertos pueden ser modificados en el archivo de configuración que suele estar
localizado en /etc/influxdb/influxdb.conf.

Por defecto, InfluxDB usa los siguientes puertos de red:

\begin{itemize}

  \item
  El puerto TCP \textbf{8086} es usado para la comunicación entre cliente y
  servidor sobre la API HTTP de InfluxDB

  \item
  El puerto TCP \textbf{8088} es usado para el servicio RPC de backup y
  restauración.


\end{itemize}


Además de estos puertos, InfluxDB ofrece múltiples plugins que pueden requerir
puertos personalizados. 
