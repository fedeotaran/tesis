\subsection{Almacenamiento}
\label{almacenamiento}

En la \autoref{metricas-y-timeseries} se han descrito las particularidades que
tienen los datos de series de tiempo. Una de estas particularidades es que
todas las series de tiempo comparten entre sí una dimensión: la dimensión de
tiempo, y por eso son útiles para correlacionar datos de varias fuentes.

Las bases de datos relacionales tradicionales pueden no ser prácticas para
manejar almacenar estos tipos de datos. Una base de datos de series de tiempo
es un sistema de \eng{software} que está optimizado para manejarlos de forma
correcta, confiable y eficiente.

Se han analizado las herramientas \gls{term:influx} y \gls{term:graphite} para
almacenar datos de este tipo.

\gls{term:influx} es una base de datos de código abierto implementada en el
lenguaje \gls{term:go} con el propósito de manejar datos de series de tiempo
con requerimientos de gran disponibilidad y alto rendimiento. \gls{term:influx}
no tiene dependencias externas.

\gls{term:influx} viene con una \gls{term:api} \gls{acro:http} integrada, lo
que permite que no sea necesario escribir ningún código del lado de servidor
para comenzar a trabajar. \cite{influxdb}

Es posible etiquetar los datos, lo que permite una consulta muy flexible. Esta
consulta se escribe en un lenguaje similar a SQL, llamado InfluxQL
\footnote{Documentación oficial:
\url{https://docs.influxdata.com/influxdb/v0.9/query_language/query_syntax/}}.

\gls{term:influx} permite contestar consultas en tiempo real, lo que significa
que cada valor se indexa a medida que llega y está disponible casi
inmediatamente para su uso.

La versión de código abierto de \gls{term:influx} no permite la configuración de
un \eng{cluster} de bases de datos. Para tenerlo es necesario actualizar a la
versión comercial de \gls{term:influx}.

\gls{term:graphite} es una herramienta de monitoreo capaz de ser ejecutada en
una amplia gama de computadoras, y también en la nube. \gls{term:graphite} es
usado para tener un seguimiento del rendimiento de sitios \eng{web},
aplicaciones, servicios de negocio y servidores en una red.

Con \gls{term:graphite} es relativamente sencillo almacenar, recuperar,
compartir y visualizar datos de tipo series de tiempo.

\gls{term:graphite} consiste en tres componentes de \eng{software}:

\begin{itemize}

  \item \textbf{\gls{term:carbon}:}
  Un servicio de alto rendimiento que recepta datos de tipo series de tiempo

  \item \textbf{\gls{term:whisper}:}
  Una base de datos sencilla para almacenar datos de tipo series de tiempo

  \item \textbf{\gls{term:graphiteweb}:}
  Una interfaz de usuario y \gls{term:api} para visualizar gráficos y tableros.

\end{itemize}

Las métricas son alimentadas a través del servicio \gls{term:carbon}, que envía
datos a las bases de datos \gls{term:whisper} para su almacenamiento a largo
plazo. Los usuarios interactúan con la interfaz de \gls{term:graphiteweb} o con
su \gls{term:api}, que a su vez consulta a \gls{term:carbon} y
\gls{term:whisper} por los datos necesarios para construir los gráficos
requeridos.


La plataforma \eng{web} de \gls{term:graphite} ofrece una variedad de estilos y
formatos de salidas \footnote{Más información:
\url{http://graphite.readthedocs.io/en/latest/overview.html}},
incluyendo imágenes, archivos separados por comas, \gls{acro:xml} y
\gls{acro:json}.

\gls{term:opentsdb} es una base de datos de series de tiempos distribuida y
escalable, escrita sobre \gls{term:hbase}. \gls{term:opentsdb} almacena, indexa
y sirve métricas recolectadas de sistemas de computadoras, como por ejemplo
redes, sistemas operativos y aplicaciones, a gran escala, y hace que estos
datos sean accesibles y graficables de forma fácil.

\gls{term:opentsdb} permite recolectar métricas de \glspl{term:host} y
aplicaciones a una taza de tiempo alta y es capaz de recolectar miles de
métricas de decenas de miles de \glspl{term:host} y aplicaciones, y almacenar
miles de millones de valores estadísticos.

\gls{term:opentsdb} es \eng{software} libre y está disponible en ambas licencias
LGPLv2.1 + GPLv3. \cite{opentsdb}

Finalmente se ha elegido \gls{term:influx} para la implementación por sobre las
demás herramientas ya que cuenta con mejor soporte para el lenguaje
\gls{term:ruby}. Además, \gls{term:influx} tiene un lenguaje de consulta similar
a SQL, lo que podría facilitar el desarrollo de consultas al personal de la
oficina.

Para ejecutar \gls{term:influx} una vez instalado, basta con correr el comando
\lstinline{service influxdb start}.

Una vez que \gls{term:influx} está instalado, es posible crear una base de
datos, insertar datos y realizar consultas usando la \gls{term:api} de
\gls{term:influx}. Incluso se puede hacer todo esto desde la consola usando el
comando \lstinline{curl}:

\begin{enumerate}
  \item{Crear una base de datos de nombre monitoreo:}
  \begin{lstlisting}
  curl -XPOST 'http://localhost:8086/query' --data-urlencode "q=CREATE DATABASE monitoreo"
  \end{lstlisting}

  \item{Insertar algunos datos:}
  \begin{lstlisting}
  curl -XPOST 'http://localhost:8086/write?db=monitoreo \
  -d 'cpu,host=server01,region=uswest load=42 1434055562000000000'

  curl -XPOST 'http://localhost:8086/write?db=monitoreo \
  -d 'cpu,host=server02,region=uswest load=78 1434055562000000000'

  curl -XPOST 'http://localhost:8086/write?db=monitoreo \
  -d 'cpu,host=server03,region=useast load=15.4 1434055562000000000'
  \end{lstlisting}


  \item{Consultar los datos:}
  \begin{lstlisting}
  curl -G http://localhost:8086/query?pretty=true --data-urlencode "db=monitoreo" \
  --data-urlencode "q=SELECT * FROM cpu WHERE host='server01' AND time < now() - 1d"

  curl -G http://localhost:8086/query?pretty=true --data-urlencode "db=monitoreo" \
  --data-urlencode "q=SELECT mean(load) FROM cpu WHERE region='uswest'"
  \end{lstlisting}
\end{enumerate}

\gls{term:influx} utiliza varios puertos de red para funcionar. Todos los
mapeos de puertos pueden ser modificados en el archivo de configuración que
suele estar localizado en \texttt{/etc/influxdb/influxdb.conf}.

Por defecto, \gls{term:influx} usa los siguientes puertos de red:

\begin{itemize}

  \item
  El puerto \gls{acro:tcp} \textbf{8086} es usado para la comunicación entre cliente y
  servidor sobre la \gls{term:api} \gls{acro:http} de \gls{term:influx}

  \item
    El puerto \gls{acro:tcp} \textbf{8088} es usado para el servicio \gls{acro:rpc} de \eng{backup} y
  restauración.

\end{itemize}

Además de estos puertos, \gls{term:influx} ofrece múltiples \eng{plugins} que
pueden requerir puertos personalizados.
