\subsection{Almacenamiento}
\label{almacenamiento}

Para poder almacenar los \eng{logs} de una forma centralizada, y hacer consultas
sobre los mismos de una forma sencilla, hemos elegido utilizar
\gls{term:elasticsearch}.

\gls{term:elasticsearch} es un servidor que provee un motor de búsqueda de
texto completo, distribuido y con capacidad de multi-tenencia con una interfaz
\eng{web} RESTful y con documentos \gls{acro:json}. \gls{term:elasticsearch}
fue desarrollado en Java y se encuentra publicado como código abierto bajo las
condiciones de la licencia Apache.

\gls{term:elasticsearch} permite almacenar, buscar y analizar datos
estructurados y no estructurados, incluyendo texto, \eng{logs} del sistema,
registros de base de datos y más. Además puede ser usado para guardar métricas
de series de tiempo proporcionadas por otras herramientas.

\gls{term:elasticsearch} permite escalabilidad y posibilidad de agregar nuevas
métricas en tiempo de ejecución, tiene facilidades para la integración con
\glspl{term:api} y herramientas como \gls{term:logstash}, y consigue un buen
rendimiento incluso al procesar grandes volúmenes de datos.

Además tiene excelente integración con la herramienta de visualización de datos
Kibana, que permite analizar datos de múltiples fuentes y correlacionar
métricas almacenadas en ElasticSearch.

\gls{term:elasticsearch} no es una base de datos relacional. Es orientada a documentos, y
tiene su propia terminología. Para facilitar la comprensión de su
funcionamiento, es importante que mencionemos los conceptos de documento e
índice en \gls{term:elasticsearch}:

\begin{itemize}
   
\item
Un \textbf{documento} es una unidad básica de información que puede ser indexada. Por
ejemplo, se puede tener un documento por un único cliente, otro documento por
un producto específico y otro documento por una orden de compras. Estos
documentos son expresados en formato \gls{acro:json}.

\item
Un \textbf{índice} es una colección de documentos que tienen características similares.
Por ejemplo, es posible tener un índice para datos de clientes, otro índice para
un catálogo de productos y otro índice para datos de órdenes de compras. Un
índice es identificado por un nombre escrito en minúsculas. Este nombre es usado
para referirse al índice al realizar operaciones de indexado, búsqueda,
actualización y eliminación sobre los documentos en el mismo.

\item
Un \textbf{tipo} en \gls{term:elasticsearch} representa una clase de documentos
similares. Un tipo consiste en un nombre y un mapeo que, como el esquema de una
base de datos, describe los campos o propiedades que los documentos de ese tipo
pueden tener.

\end{itemize}

\gls{term:elasticsearch} provee una \gls{term:api} RESTful \gls{acro:json} sobre
\gls{acro:http} para realizar todas las operaciones importantes. Por defecto,
\gls{term:elasticsearch} corre sobre el puerto 9200. Es posible comunicarse con
esta \gls{term:api} mediante clientes \eng{web}, o también usando el comando
\texttt{curl} desde una terminal.

\gls{term:elasticsearch} provee clientes oficiales para muchos lenguajes, como
Groovy, JavaScript, .NET, PHP, Perl, Python y \gls{term:ruby}, y hay numerosos
clientes e integraciones provistos por la comunidad, que pueden ser encontrados
en el sitio \eng{web} de \gls{term:elasticsearch}.

Por ejemplo, si quisiera indexar un documento por cada empleado, de tal forma
que cada documento sea de tipo empleado y ese tipo exista en el índice de la
biblioteca, basta con llamar a la \gls{term:api} de la siguiente manera:

\begin{lstlisting}

PUT /biblioteca/empleado/1
{
    "nombre" :   "Juan",
    "apellido" : "Gomez",
    "edad" :      25,
    "areas" :     [ "encargado" ]
}

\end{lstlisting}


Para instalar \gls{term:elasticsearch} es necesario contar con Java en el
equipo. Una vez configurado Java, simplemente es necesario descargar y
ejecutar \gls{term:elasticsearch}. Para descargar y ejecutar
\gls{term:elasticsearch} 5.1, podemos correr los siguientes comandos:

\begin{lstlisting}

curl -L -O https://artifacts.elastic.co/downloads/elasticsearch/elasticsearch-5.1.1.tar.gz
tar -xvf elasticsearch-5.1.1.tar.gz
cd elasticsearch-5.1.1/bin
./elasticsearch

\end{lstlisting}

Para nuestra solución, hemos elegido instalar \gls{term:elasticsearch} en un
servidor aparte. Esto nos da por lo menos dos ventajas:

En primer lugar, guardar los datos de los \eng{logs} en un entorno diferente al
que corren las aplicaciones, nos permite tener disponibilidad de los datos en
caso de que ocurra un error que afecte a la infraestructura completa.

Además, al tener una sola instancia separada, se logra centralizar los datos de
los \eng{logs} y se tiene una única fuente que consultar.

Antes de continuar con la siguiente sección, creemos importante comentar que en
una solución centralizada es importante poder identificar la fuente de cada uno
de los \eng{logs}.
