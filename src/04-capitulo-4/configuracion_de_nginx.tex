\subsection{Configuración de Nginx} \label{configuracion_de_nginx}

\gls{term:nginx} es una herramienta multiplataforma que funciona como servidor
\eng{web} o proxy inverso ligero\footnote{Cf.
\url{https://www.nginx.com/resources/admin-guide/reverse-proxy/}} de alto
rendimiento. Es software libre y de código abierto, y está licenciado bajo la
licencia \gls{acro:bsd} simplificada.

Es posible usar \gls{term:nginx} como un balanceador de carga \gls{acro:http}
para distribuir tráfico entre varios servidores de aplicaciones y para aumentar
el rendimiento, escalabilidad y confiabilidad de aplicaciones
\eng{web} \footnote{Cf.
\url{http://nginx.org/en/docs/http/load_balancing.html}}.

\gls{term:nginx} es conocido por su alta performance, estabilidad, rico
conjunto de utilidades, configuración sencilla y bajo consumo de recursos.

La infraestructura de la oficina a partir de contenedores de \gls{term:docker},
nos otorga casi gratuitamente la funcionalidad de balanceo de carga. En este
trabajo usaremos \gls{term:nginx} simplemente como un medio confiable y
eficiente para obtener información de todas las solicitudes que reciban las
aplicaciones \eng{web}.

De esta forma, podremos obtener información precisa sobre el tiempo de
respuesta de las aplicaciones desde el momento en que se recibe la petición
hasta que la solicitud es respondida.

\gls{term:nginx} escribe información sobre solicitudes de clientes en el access
log luego de que la petición \eng{web} es procesada. Por defecto, este
archivo se ubica en la ruta \texttt{logs/access.log}, y la información es
impresa en el mismo en un formato predefinido.

Para sobrescribir la configuración por defecto, se puede usar la directiva
\texttt{log\_format}, que modifica el formato de los mensajes logueados así
como la directiva \texttt{access\_log}, que especifica la ruta del archivo de
logs y el formato de los mismos.

Para utilizar \gls{term:nginx} utilizaremos la imagen de \gls{term:docker}
oficial [\url{https://hub.docker.com/_/nginx/}] de la herramienta, que se
encuentra publicada en DockerHub[referencia de que es dockerhub].

Dado que nuestra propuesta es utilizar \gls{term:docker} como entorno virtual
en donde corren los distintos servicios, debemos tener en cuenta las
particularidades que tiene esta herramienta para poder obtener los logs de la
aplicación en ejecución.

Al igual que para las aplicaciones \gls{term:ror}, para obtener los registros
del servicio \gls{term:nginx} que se encuentra en ejecución dentro de un
contenedor \gls{term:docker}, es necesario hacer uso de uno de sus drivers.
También en este caso utilizaremos la herramienta \gls{term:fluentd} para
recolectar los logs, por lo que usaremos el driver que lleva el mismo nombre.
Por defecto, la imagen de \gls{term:nginx} es configurada para enviar los logs
principales de acceso y error de \gls{term:nginx} al colector de logs de
\gls{term:docker}. Esto se logra enlazando los logs con la salida estándar
\gls{acro:stdout} y la salida de errores \gls{acro:stderr}.

Con esto obtenemos la configuración necesaria para obtener la información de
los logs del servicio y dejarlos disponibles para la herramienta de
recolección.
