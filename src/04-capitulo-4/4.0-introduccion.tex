Para implementar la nueva infraestructura de monitoreo, y con el objetivo de
darles uso real a los registros de las aplicaciones, nuestro primer paso será
intentar utilizar la información que brindan los logs de forma más intensiva.

En \gls{term:ror}, los logs son complejos. Cada registro puede ocupar varias
líneas, y cada tipo de log tiene su propio formato. En la
\autoref{configuracion_de_las_aplicaciones} explicaremos cómo extraer valiosa
información de los logs de las aplicaciones del laboratorio.

Utilizaremos los logs de \gls{term:nginx} para recolectar datos precisos sobre
la velocidad de respuesta de las aplicaciones ante solicitudes \eng{web}.  En
la \autoref{configuracion_de_nginx} explicaremos cómo configurar la herramienta
para que la información recolectada sea útil en un ambiente de trabajo con
contenedores de \gls{term:docker}.

Para poder hacer uso de la información de los registros de ambas fuentes,
necesitaremos algún tipo de herramienta de almacenamiento y búsqueda. En la
\autoref{almacenamiento} describiremos cómo configurar \gls{term:elasticsearch}
como motor de búsqueda y análisis de logs.

En la \autoref{unificacion_y_recoleccion} explicaremos cómo unificar los logs
de diversas fuentes y cómo centralizarlos en un la base de datos descripta en
el capítulo anterior.  Utilizaremos \gls{term:fluentd} para lograr que las
aplicaciones y las instancias de \gls{term:nginx} se comuniquen correctamente
con \gls{term:elasticsearch}, y mostraremos cómo configurarlo.

Una vez que tengamos todo el sistema de recolección, unificación,
centralización y almacenamiento de logs configurado, demostraremos cómo obtener
información útil a través de una \gls{term:api} y utilizando un lenguaje de
consultas. En la \autoref{consultas} daremos algunos ejemplos de uso del
sistema construido en funcionamiento.

