\subsection{Configuración de las aplicaciones}
\label{configuracion_de_las_aplicaciones}

Para poder hacer uso de los logs de nuestras aplicaciones, debimos tener en
cuenta que la oficina se encuentra en un proceso de migración de
infraestructura a contenedores de \gls{term:docker}.

La manera recomendada en la documentación de \gls{term:docker} para realizar
logging, es imprimiendo los logs de nuestras aplicaciones en el
\gls{acro:stdout} \footnote{Cf.
\url{https://docs.docker.com/engine/admin/logging/view_container_logs/}}.  Es
por ello que debimos configurar los logs de nuestras aplicaciones para que sean
escritos en la salida estándar. Utilizar esta estrategia nos permite contar
con información del contenedor en ejecución, como lo son su identificador y su
nombre.

\gls{term:docker} tiene soporte para múltiples mecanismos de logging. Al
iniciar el daemon de \gls{term:docker} se puede usar la bandera
\texttt{--log-opt NAME=VALUE} para especificar el driver para realizar logging
que usará por defecto el contenedor.

Para este trabajo utilizaremos otra forma de configurar el driver, que es a
través de la opción \texttt{log-driver=VALUE} durante el comando
\gls{term:docker}
run\footnote{\url{https://docs.docker.com/engine/admin/logging/overview/}}.

Una vez hecho esto, el siguiente paso es configurar nuestras aplicaciones para
que envíen sus logs a la salida estándar.

Por defecto \gls{term:ror} mantiene un archivo separado para los logs por cada
ambiente de ejecución, y para hacer uso de la información de los logs utiliza
la clase \texttt{ActiveSupport::Logger}. Aunque \gls{term:ror} viene con una
configuración para el manejo de log predeterminada, puede puede cambiarse la
misma especificando un logger alternativo en un archivo de configuración de la
siguiente forma:

\begin{lstlisting}

config.logger = SomeLogger

\end{lstlisting}

Podemos lograr que \gls{term:docker} capture correctamente los logs de
\gls{term:ror} agregando la siguiente especificación al archivo
config/application.rb de la aplicación:

\begin{lstlisting}

config.logger = ActiveSupport::Logger.new(STDOUT)

\end{lstlisting}

Los logs de \gls{term:ror} tienen algunas características que hacen que sea
difícil trabajar con ellos. En primer lugar, no siguen un formato estándar.
Esto hace que para distintos tipos de logs sea necesario usar diferentes
estrategias para recuperar la información valiosa. En segundo lugar, un solo
log puede ocupar varias líneas, lo que hace complejo identificar dónde termina
un log y dónde empieza el siguiente.

Para resolver estos problemas utilizamos una gema llamada lograge\footnote{Cf.
\url{https://github.com/roidrage/lograge}}. Esta gema se encarga de transformar
complejos logs multilínea de \gls{term:ror} en simples logs con datos
importantes bien identificados de una sola línea.

Por ejemplo, un log de \gls{term:ror} se ve de la siguiente manera:

\begin{lstlisting}

Started GET "/" for 126.0.0.1 at 2012-03-10 14:28:14 +0100
Processing by HomeController#index as HTML
  Rendered text template within layouts/application (0.0ms)
  Rendered layouts/_assets.html.erb (2.0ms)
  Rendered layouts/_top.html.erb (2.6ms)
  Rendered layouts/_about.html.erb (0.3ms)
  Rendered layouts/_google_analytics.html.erb (0.4ms)
Completed 200 OK in 79ms (Views: 78.8ms | ActiveRecord: 0.0ms)

\end{lstlisting}

Usando lograge, el mismo log se transforma en:

\begin{lstlisting}

method=GET path=/jobs/833552.json format=json controller=JobsController  action=show status=200 duration=58.33 view=40.43 db=15.26

\end{lstlisting}

Para configurar lograge en la aplicación \gls{term:ror}, debe incluirse la gema
con el mismo nombre a la lista de dependencias de la aplicación. Para hacerlo
basta con agregar la siguiente línea de código al archivo \texttt{Gemfile}:

\begin{lstlisting}

gem 'lograge'

\end{lstlisting}

Luego debe agregarse en el archivo \texttt{config/application.rb} la siguiente
especificación:

\begin{lstlisting}

class Application < Rails::Application
  config.logger = ActiveSupport::Logger.new(STDOUT)
  config.log_formatter = nil
  config.log_level = :info

  config.lograge.enabled = true
  config.lograge.formatter = Lograge::Formatters::Json.new
  config.lograge.custom_options = lambda do |event|
    {
      app_timestamp: event.time
    }
  end
end

\end{lstlisting}

Con estas configuraciones ya contamos con una aplicación \gls{term:ror} que
envía la información de sus logs a la salida estándar en un formato más fácil
de procesar posteriormente.
