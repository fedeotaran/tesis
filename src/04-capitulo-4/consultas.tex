\subsection{Consultas}
\label{consultas}

Una vez que tenemos todo configurado, podemos comenzar a realizar consultas
sobre la base de datos \gls{term:elasticsearch} donde están los logs. A través
del comando curl podemos solicitar los datos de los índices creados en la base
de datos por nuestras herramientas:

\begin{lstlisting}

curl localhost:9200/_cat/indices?v

\end{lstlisting}

\begin{lstlisting}

health status index                 uuid                   pri rep docs.count docs.deleted store.size pri.store.size
yellow open   nginx-logs-2017.03.22 fNCeXckpQwyWtr5dntS9aQ   5   1          4            0       51kb           51kb
yellow open   rails-logs-2017.03.22 nh9ID9o3TwmrxYmBOven8Q   5   1         46            0     87.3kb         87.3kb
yellow open   .kibana               sHMjKRIWRuKrtZlZ9b31IQ   1   1          2            0     10.2kb         10.2kb

\end{lstlisting}

En la respuesta de la consulta se pueden visualizar tres índices:
\texttt{.kibana}, \text{rails-logs-2017.02.21} y
\texttt{nginx-logs-2017.02.21}.

Si quisiéramos visualizar la configuración de uno de estos índices, alcanzaría
con hacer una consulta como la siguiente:

\begin{lstlisting}

curl localhost:9200/rails-logs-2017.02.21

\end{lstlisting}

\begin{lstlisting}

{
  "rails-logs-2017.03.22": {
    "aliases": {},
      "mappings": {
        "fluentd": {
          "properties": {
            "@timestamp": {
              "type": "date"
            },
            "action": {
              "fields": {
                "keyword": {
                  "ignore_above": 256,
                  "type": "keyword"
                }
              },
              "type": "text"
            },
            "app_timestamp": {
              "fields": {
                "keyword": {
                  "ignore_above": 256,
                  "type": "keyword"
                }
              },
              "type": "text"
            },
            "container_id": {
              "fields": {
                "keyword": {
                  "ignore_above": 256,
                  "type": "keyword"
                }
              },
              "type": "text"
            },
            "container_name": {
              "fields": {
                "keyword": {
                  "ignore_above": 256,
                  "type": "keyword"
                }
              },
              "type": "text"
            },
            "controller": {
              "fields": {
                "keyword": {
                  "ignore_above": 256,
                  "type": "keyword"
                }
              },
              "type": "text"
            },
            "db": {
              "type": "float"
            },
            "duration": {
              "type": "float"
            },
            "format": {
              "fields": {
                "keyword": {
                  "ignore_above": 256,
                  "type": "keyword"
                }
              },
              "type": "text"
            },
            "log": {
              "fields": {
                "keyword": {
                  "ignore_above": 256,
                  "type": "keyword"
                }
              },
              "type": "text"
            },
            "method": {
              "fields": {
                "keyword": {
                  "ignore_above": 256,
                  "type": "keyword"
                }
              },
              "type": "text"
            },
            "path": {
              "fields": {
                "keyword": {
                  "ignore_above": 256,
                  "type": "keyword"
                }
              },
              "type": "text"
            },
            "source": {
              "fields": {
                "keyword": {
                  "ignore_above": 256,
                  "type": "keyword"
                }
              },
              "type": "text"
            },
            "status": {
              "type": "long"
            },
            "view": {
              "type": "float"
            }
          }
        }
      },

      .
      .
      .

  }
}

\end{lstlisting}

En el resultado de la consulta se puede ver la estructura de los registros y la
composición del índice que almacena los logs de las aplicaciones rails.

Para visualizar todos los documentos dentro de un índice, se puede ejecutar la
siguiente instrucción:

\begin{lstlisting}

curl localhost:9200/nginx-logs-2017.02.21/_search

\end{lstlisting}

\begin{lstlisting}

{
    "_shards": {
        "failed": 0,
        "successful": 5,
        "total": 5
    },
    "hits": {
        "hits": [
            {
                "_id": "AVpiotVhd2D1TjttLYvK",
                "_index": "nginx-logs-2017.02.21",
                "_score": 1.0,
                "_source": {
                    "@timestamp": "2017-02-21T21:45:20+00:00",
                    "container_id": "e7e13c19bf035d28f8d5ed087b7fd60aca6c82e8dce6223ccd8b5a03e6a6240b",
                    "container_name": "/logger_nginx_1",
                    "log": "172.23.0.1 - - [21/Feb/2017:21:45:20 +0000] \"GET / HTTP/1.1\" 304 0 \"-\" \"Mozilla/5.0 (X11; Linux x86_64) AppleWebKit/537.36 (KHTML, like Gecko) Chrome/56.0.2924.87 Safari/537.36\" \"-\"",
                    "referer": "-",
                    "remote_addr": "172.23.0.1",
                    "remote_user": "-",
                    "request_http_protocol": "HTTP/1.1",
                    "request_size": "0",
                    "request_type": "GET",
                    "request_url": "/",
                    "source": "stdout",
                    "status_code": "304",
                    "user_agent": "Mozilla/5.0 (X11; Linux x86_64) AppleWebKit/537.36 (KHTML, like Gecko) Chrome/56.0.2924.87 Safari/537.36"
                },
                "_type": "fluentd"
            },
           .
           .
           .
        ],
        "max_score": 1.0,
        "total": 10
    },
    "timed_out": false,
    "took": 2
}

\end{lstlisting}

Esta consulta nos retorna información sobre todos los documentos en el índice
de requerimientos \eng{web} que han pasado por \gls{term:nginx}. Entre
otros datos, se pueden ver el identificador y nombre del contenedor, el código
de estado \gls{acro:http}, el cliente \eng{web} utilizado y la marca de
tiempo.

\gls{term:fluentd} nos ha permitido analizar el contenido del mensaje de los
logs y separar aquellos campos importantes. Esto nos permitirá realizar
búsquedas fácilmente.

A lo largo de este capítulo hemos conseguido configurar varias herramientas
para unificar los logs de diferentes fuentes en una base de datos centralizada,
y utilizar un motor de búsquedas para obtener información importante de ellos.

Los logs pueden ser muy útiles, pero no son la única fuente de datos de la que
se puede obtener información valiosa. En el capítulo siguiente analizaremos
cómo recolectar datos de diversas fuentes generados en tiempo real.

