\subsection{Configuración de las aplicaciones}
\label{configuracion_de_las_aplicaciones}

Para poder hacer uso de los \eng{logs} de nuestras aplicaciones, debimos tener en
cuenta que la oficina se encuentra en un proceso de migración de
infraestructura a \glspl{term:contenedor} de \gls{term:docker}.

La manera recomendada en la documentación de \gls{term:docker} para realizar
logging, es imprimiendo los \eng{logs} de nuestras aplicaciones en \gls{acro:stdout}
\footnote{Cf. \url{https://docs.docker.com/engine/admin/logging/view_container_logs/}}.
Es por ello que debimos configurar los \eng{logs} de nuestras aplicaciones para que sean
escritos en la salida estándar. Utilizar esta estrategia nos permite contar
con información del contenedor en ejecución, como lo son su identificador y su
nombre.

\gls{term:docker} tiene soporte para múltiples mecanismos de logging. Al
iniciar el daemon de \gls{term:docker} se puede usar la bandera
\texttt{--log-opt NAME=VALUE} para especificar el driver para realizar logging
que usará por defecto el contenedor.

Para este trabajo utilizaremos otra forma de configurar el driver, que es a
través de la especificación de la opción \texttt{log-driver=VALUE} durante el
comando \lstinline{docker run}.
\footnote{\url{https://docs.docker.com/engine/admin/logging/overview/}}.

Una vez hecho esto, el siguiente paso es configurar nuestras aplicaciones para
que envíen sus \eng{logs} a la salida estándar.

Por defecto \gls{term:ror} mantiene un archivo separado para los \eng{logs} por cada
ambiente de ejecución, y para hacer uso de la información de los \eng{logs} utiliza
de forma predeterminada la clase \texttt{ActiveSupport::Logger}.

En \gls{term:ror} es posible cambiar la configuración de la clase que maneja los
\eng{logs} especificando la alternativa en un archivo de configuración de la siguiente manera:

\begin{lstlisting}

config.logger = SomeLogger

\end{lstlisting}

Para que \gls{term:docker} capture correctamente los \eng{logs} de \gls{term:ror}
podemos agregar la siguiente especificación al archivo
\texttt{config/application.rb} de la aplicación:

\begin{lstlisting}

config.logger = ActiveSupport::Logger.new(STDOUT)

\end{lstlisting}

Los \eng{logs} de \gls{term:ror} tienen algunas características que hacen que sea
difícil trabajar con ellos. En primer lugar, no siguen un formato estándar.
Esto hace que para distintos tipos de \eng{logs} sea necesario usar diferentes
estrategias para recuperar información valiosa. En segundo lugar, un solo
\eng{log} puede ocupar varias líneas, lo que hace complejo identificar dónde termina
un \eng{log} y dónde empieza el siguiente.

Para resolver estos problemas utilizamos una \gls{term:gema} llamada \gls{term:lograge}
\footnote{Cf. \url{https://github.com/roidrage/lograge}}. Esta \gls{term:gema} se encarga
de transformar complejos \eng{logs} multilínea de \gls{term:ror} en simples \eng{logs} con
datos importantes bien identificados de una sola línea.

Por ejemplo, un \eng{log} de \gls{term:ror}, creado al acceder a la página \eng{home}
de una de las aplicaciones del laboratorio se ve de la siguiente manera:

\begin{lstlisting}

Started GET "/" for 126.0.0.1 at 2012-03-10 14:28:14 +0100
Processing by HomeController#index as HTML
  Rendered text template within layouts/application (0.0ms)
  Rendered layouts/_assets.html.erb (2.0ms)
  Rendered layouts/_top.html.erb (2.6ms)
  Rendered layouts/_about.html.erb (0.3ms)
  Rendered layouts/_google_analytics.html.erb (0.4ms)
Completed 200 OK in 79ms (Views: 78.8ms | ActiveRecord: 0.0ms)

\end{lstlisting}

Usando \gls{term:lograge}, el mismo \eng{log} es transformado en:

\begin{lstlisting}

method=GET path=/jobs/833552.json format=json controller=JobsController action=show status=200 duration=58.33 view=40.43 db=15.26

\end{lstlisting}

Para configurar \gls{term:lograge} en \gls{term:ror}, debe incluirse la \gls{term:gema} con
el mismo nombre a la lista de dependencias de la aplicación. Para hacerlo basta
con agregar la siguiente línea de código al archivo \texttt{Gemfile}:

\begin{lstlisting}

gem 'lograge'

\end{lstlisting}

Luego debe agregarse en el archivo \texttt{config/application.rb} la siguiente
especificación:

\begin{lstlisting}

class Application < Rails::Application
  config.logger = ActiveSupport::Logger.new(STDOUT)
  config.log_formatter = nil
  config.log_level = :info

  config.lograge.enabled = true
  config.lograge.formatter = Lograge::Formatters::Json.new
  config.lograge.custom_options = lambda do |event|
    {
      app_timestamp: event.time
    }
  end
end

\end{lstlisting}

Con estas configuraciones ya contamos con una aplicación \gls{term:ror} que
envía la información de sus \eng{logs} a la salida estándar en un formato cuyo
procesamiento se puede realizar de forma más sencilla.
