La visualización de datos es el proceso de interpretación, contrastación y
comparación de datos que permite un conocimiento en profundidad y detalle de
los mismos de tal forma que se transformen en información comprensible para
las personas.

En la \autoref{herramientas-de-visualizacion} se explica qué son las
herramientas de visualización de datos y se describen las herramientas
\gls{term:kibana} y \gls{term:grafana}.

En la \autoref{configuracion-de-kibana} se explica como configurar
\gls{term:kibana} para realizar consultas en nuestra instancia de
\gls{term:elasticsearch}. Además se muestra cómo crear gráficos a partir de
estos datos y de qué forma explorar los \eng{logs} desde el cliente \eng{web}
de \gls{term:kibana}.

En la \autoref{configuracion-de-grafana} se muestra cómo configurar
\gls{term:grafana} para que trabaje de forma correcta con \gls{term:influx}.
Además se configura un \gls{term:datasource} vinculado con la instancia de
\gls{term:influx} y se crean tableros con gráficos que hagan uso de esta
información.
