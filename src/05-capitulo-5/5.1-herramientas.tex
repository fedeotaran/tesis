\subsection{Herramientas}
\label{herramientas-de-visualizacion}

Las herramientas de \eng{software} de visualización de datos son aquellas que se
ocupan de mostrar datos de distintas fuentes en forma de gráficos, de forma que
los usuarios puedan descubrir patrones y entender la información de forma
sencilla.

Existen varias herramientas informáticas para visualizar datos. En esta sección
se describirán brevemente las herramientas \gls{term:kibana} y
\gls{term:grafana}.

\begin{itemize}

\item
\textbf{\gls{term:kibana}} es una plataforma de análisis y visualización de
código abierto diseñada para trabajar con \gls{term:elasticsearch}. Se lo puede
usar para buscar, observar e interactuar con datos almacenados en índices de
esta herramienta.

Con \gls{term:kibana} es posible realizar análisis de datos complejos y
visualizar datos en una variedad de gráficos, tablas y mapas.
\autoref{fig:kibana}

\begin{figure}
  \includegraphics[width=\linewidth]{src/images/05-capitulo-5/kibana.jpg}
  \caption{Tablero de \gls{term:kibana}}
  \label{fig:kibana}
\end{figure}

\gls{term:kibana} cuenta con una interfaz basada en el navegador, que permite
crear y compartir tableros de control dinámicos que muestran cambios en tiempo
real a partir de consultas a \gls{term:elasticsearch}.

\item
\textbf{\gls{term:grafana}} es un tablero y componedor de gráficos de propósito
general y de código abierto que corre como una aplicación \eng{web}. Es
comúnmente usado para visualizar datos de series de tiempo para infraestructuras
en la \eng{web} y analíticas de aplicación, pero también es usado en sensores
industriales, automatización de viviendas, medición del clima y control de
procesos. \autoref{fig:grafana}

\begin{figure}
  \includegraphics[width=\linewidth]{src/images/05-capitulo-5/grafana.png}
  \caption{Tablero de \gls{term:grafana}}
  \label{fig:grafana}
\end{figure}

\gls{term:grafana} permite una sencilla extensibilidad y variedad de paneles,
con ricas opciones de visualización, y tiene soporte para las fuentes de datos
de series de tiempo más populares, incluyendo \gls{term:influx} y
\gls{term:elasticsearch}.

\end{itemize}

Luego de analizar estas opciones, se consensuó que la herramienta de
visualización más completa entre ambas era \gls{term:grafana}.

Pero \gls{term:kibana} es un \eng{frontend} de \gls{term:elasticsearch} y está
preparado especialmente para realizar consultas sobre esta base de datos de
forma simple. Con \gls{term:kibana} es posible hacer consultas directas sobre
los registros de \eng{logs} tal cual fueron recuperados e indexados.

Es por esto que se decidió usar ambas herramientas: \gls{term:kibana} para
consultar las base de datos de \gls{term:elasticsearch}, y hacer consultas
acerca de los \eng{logs}, y \gls{term:grafana} para comunicarse con la instancia
de \gls{term:influx} y visualizar datos generados en tiempo real.

Estas herramientas están en constante crecimiento, por lo que no se descarta que
en un futuro \gls{term:grafana} incorpore funcionalidades que se encuentran
presentes en \gls{term:kibana}, y viceversa.
