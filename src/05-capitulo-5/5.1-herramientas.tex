\subsection{Herramientas}
\label{herramientas-de-visualizacion}

Las herramientas de software de visualización de datos son aquellas que ocupan de mostrar datos de distintas fuentes en forma de gráficos, de forma que los usuarios puedan descubrir patrones y entender la información de forma sencilla.

Existen varias herramientas informáticas para visualizar datos. En esta sección describiremos brevemente a las herramientas Kibana y Grafana.

Kibana es una plataforma de análisis y visualización de código abierto diseñada para trabajar con Elasticsearch. Se lo puede usar para buscar, observar e interactuar con datos almacenados en índices de Elasticsearch.

Con Kibana es posible realizar análisis de datos complejos y visualizar datos en una variedad de gráficos, tablas y mapas. \autoref{fig:kibana}


\begin{figure}
  \includegraphics[width=\linewidth]{src/images/05-capitulo-5/kibana.jpg}
  \caption{Tablero de Kibana}
  \label{fig:kibana}
\end{figure}

Kibana cuenta con una interfaz basada en el navegador, que permite crear y compartir tableros de control dinámicos que muestran los cambios a consultas de Elasticsearch en tiempo real.

Como hemos mencionado en la teoría, el contar con formas visuales de representar información permite a las personas entender grandes volúmenes de datos de forma más sencilla.

Grafana es un tablero y componedor de gráficos de propósito general y de código abierto que corre como una aplicación web. Es comúnmente usado para visualizar datos de series de tiempo para infraestructuras en la web y analíticas de aplicación, pero también es usado en sensores industriales, automatización de viviendas, medición del clima y control de procesos.\autoref{fig:grafana}

\begin{figure}
  \includegraphics[width=\linewidth]{src/images/05-capitulo-5/grafana.png}
  \caption{Tablero de Grafana}
  \label{fig:grafana}
\end{figure}

Grafana permite una sencilla extensibilidad y variedad de paneles, con ricas opciones de visualización, y tiene soporte para las fuentes de datos de series de tiempo más populares, incluyendo InfluxDB y ElasticSearch.

Luego de analizar estas opciones, nos pareció que la herramienta de visualización más completa entre ambas era Grafana.

Pero Kibana es un front-end de elasticsearch y está preparado especialmente para realizar consultas sobre esta base de datos de forma simple. Con Kibana es posible hacer consultas directas sobre los registros de logs tal cual fueron recuperados e indexados.

Es por esto que decidimos usar ambas herramientas: Kibana para consultar las base de datos de Elasticsearch, y hacer consultas acerca de los logs, y Grafana para comunicarse con nuestra instancia de InfluxDB y visualizar datos generados en tiempo real.

Creemos importante destacar que estas herramientas están en constante crecimiento, por lo que no descartamos que en un futuro Grafana incorpore funcionalidades que se encuentren presentes en Kibana, y viceversa.
