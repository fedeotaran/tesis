\subsection{Monitoreo del sistema de alertas}
\label{del-sistema-de-alertas}

Los sistemas de alertas pueden generar información muy importante. Tener un
seguimiento de las alertas en sistemas de monitoreo propios podría permitir por ejemplo realizar consultas sobre la cantidad de alertas generadas por período de
tiempo.

Tener tableros de control que muestren este tipo de información, podría ser
útil para jefes de proyecto. Las alertas generadas podrían ser una buena
métrica para ver la estabilidad de la aplicación. Por ejemplo se podría
correlacionar la generación de alertas con la cantidad de memoria, o el
procesamiento de las CPUs a lo largo del tiempo.

Podrían utilizarse algunas herramientas que se han mencionado a lo
largo de la tesis para tomar información de \gls{term:kapacitor} y con esa
información crear tableros en tiempo real que permitan mantener un seguimiento
de la generación de alertas durante la evolución de las aplicaciones.
