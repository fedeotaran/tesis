\subsection{Monitoreo que aprenda sobre sí mismo}
\label{que-aprenda}

Como hemos mencionado anteriormente el monitoreo es un aspecto en constante
crecimiento. Cada vez se van aplicando más conceptos y se mejoran las prácticas
para obtener resultados mejores y más rápidos.

El uso de algoritmos de \eng{machine learning} en el monitoreo es uno de los
aspecto que está tomando más fuerza. Como dice Allois Reitbauer “La
construcción de sistemas de monitoreo de auto-aprendizaje ayuda a los equipos
de operaciones a concentrarse en sus tareas centrales en lugar de intentar
interpretar un tablero de gráficos. El monitoreo inteligente también está en el
núcleo del movimiento
\gls{term:devops}”\cite[Prefacio]{monitoreo:anomaly_detection_for_monitoring}

El buen uso de estas técnicas puede traer grandes beneficios a las
organizaciones. En sistemas bien implementados este tipo de tecnología puede
identificar los problemas rápidamente y cada vez mejor, incluso adelantarse a
que el problema ocurra.

