\subsection{Monitoreo que aprenda sobre sí mismo}
\label{que-aprenda}

Como se ha mencionado anteriormente, el monitoreo es una rama de investigación
que está en constante crecimiento. El uso de algoritmos de
\eng{machine learning} en el monitoreo es uno de los aspectos que está tomando
más fuerza.

Allois Reitbauer explica: “La construcción de sistemas de monitoreo de
auto-aprendizaje ayuda a los equipos de operaciones a concentrarse en sus tareas
centrales en lugar de intentar interpretar un tablero de gráficos. El monitoreo
inteligente también está en el núcleo del movimiento \gls{term:devops}”
\cite[Prefacio]{monitoreo:anomaly_detection_for_monitoring}

El uso de técnicas de auto-aprendizaje es un posible trabajo futuro.
Estas técnicas pueden traer grandes beneficios a las organizaciones,
ayudándolas a identificar problemas rápidamente, realizando esta tarea con mayor
precisión en cada iteración e incluso adelantándose a que los problemas ocurran.
